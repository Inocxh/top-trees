\chapter{Examples of problems and user functions}
\label{chap:Problems}

In this section there is a list of several problems which could be solved by
Top Trees with small time complexity.

\section{Finding distance between two vertices}

This problem was originally solved in {\I A Data Structure for Dynamic Trees}
\cite{DSforDynamicTrees} by Sleator and Tarjan in 1983 and then it was adapted
for Top Trees by Alstrup, Holm, Lichtenberg and Thorup in \cite{TopTrees}.

{\bf Theorem:} Lets have dynamic collection of weighted trees with link and cut
operations. We could find length of the path between any two vertices (or find
that they are not connected) in $\O(\log N)$ time.

\medskip\noindent
{\bf Proof:} We will maintain length of the cluster path in every cluster.

\begin{itemize}

\item $\Create$ creates cluster with length equivalent to the length of the
underlying edge.

\item $\Join$ of clusters $C_1$ and $C_2$ into $C$ depends on the type of the $C$:
	\begin{itemize}
	\item If $C$ is a compress cluster of $C_1$ and $C_2$:
	Set length of the $C$'s path as sum of the lengths of $C_1$ and $C_2$.
	\item If $C$ is a rake cluster and $C_1$ is raked onto $C_2$:
	Set length of the $C$'s path as length of the $C_2$'s path (and vice
	versa if $C_2$ is raked onto $C_1$).
	\end{itemize}

\item $\Split$ and $\Destroy$ does nothing.

\end{itemize}

After that we could easily get length of the $(u,v)$-path by calling
$\Expose(u,v)$ and reading length from this cluster. Because operations
in user functions are in constant time and \Expose{} takes $\O(\log N)$
operations, we could answer on any such question in $\O(\log N)$ time.

%%%%%%%%%%%%%%%%%%%%%%%%%%%%%%%%%%%%%%%%%%%%%%%%%%%%%%%%%%%%%%%%%%%%%%%%%%%%%%%%

\section{Maximum edge weight between given vertices with interval update}
\label{sec:maximum_edge_weight}

Similarly to the previous problem this problem was originally solved in
\cite{DSforDynamicTrees} and then it was adapted for Top Trees  in
\cite{TopTrees}.

{\bf Theorem:} Let's have dynamic collection of weighted trees with operation of linking,
cutting, updating edge weight and updating edge weights on given path. We could
find maximum edge weight between any two vertices (or find that they are not
connected) in $\O(\log N)$ time.

\medskip\noindent
{\bf Proof:} We can maintain $w_{max}$ in each cluster as maximum weight on this
cluster's path and $w_{extra}$ as weight added to each edge on this path.

\begin{itemize}

\item $\Create$ creates cluster with $w_{max}=w(e)$ where $e$ is the edge for
which is this cluster created. There is no extra weight yet, so $w_{extra}=0$.

\item $\Join$ of clusters $C_1$ and $C_2$ into $C$ depends on the type of the $C$:
	\begin{itemize}
	\item If $C$ is a compress cluster of $C_1$ and $C_2$:
	Set $w_{max}$ as maximum of $w_{max}$ from clusters $C_1$ and $C_2$.
	There is no extra weight, so $w_{extra}=0$.
	\item If $C$ is a rake cluster and $C_1$ is raked onto $C_2$:
	Copy $w_{max}$ from the $C_2$. There is no extra weigh, so $w_{extra}=0$.
	\end{itemize}

\item $\Split$ have to distribute $w_{extra}$ to the children. For $C_1, C_2$,  children
of splitted $C$ will operations depend on the type of the $C$:
	\begin{itemize}
	\item If $C$ is a compress cluster of $C_1$ and $C_2$:
		\begin{itemize}[$\circ$]
		\item $w_{extra}(C_i)=w_{extra}(C_i)+w_{extra}(C)$ for $i\in{1,2}$
		\item $w_{max}(C_i)=w_{max}(C_i)+w_{extra}(C)$ for $i\in{1,2}$
		\end{itemize}
	\item If $C$ is a rake cluster and $C_1$ is raked onto $C_2$:
	Apply above operation only for $C_2$ (and vice versa only for $C_1$ if $C_2$
	is raked onto $C_1$).
	\end{itemize}


\item $\Destroy$ sets weight of the underlying edge: $w(e)=w_{max}+w_{extra}$.

\end{itemize}

Then we could just call $\Expose(u,v)$ and read $w_{max}$ or add both to
$w_{extra}$ and $w_{max}$ of the root cluster representing the $(u,v)$-path.
Everything in time complexity of an {\sc Expose} operation, which is $\O(\log
N)$.

%%%%%%%%%%%%%%%%%%%%%%%%%%%%%%%%%%%%%%%%%%%%%%%%%%%%%%%%%%%%%%%%%%%%%%%%%%%%%%%%

%\section{Diameter and center of the trees}

%%%%%%%%%%%%%%%%%%%%%%%%%%%%%%%%%%%%%%%%%%%%%%%%%%%%%%%%%%%%%%%%%%%%%%%%%%%%%%%%

\section{Edge 2-connectivity}
\label{sec:edge_2_connectivity}

This is an example of more complex problem, where is the Top tree structure used
as only part of the whole algorithm. It was introduced by Holm, Lichtenberg and
Thorup in 2001 in \cite{PolylogarithmicAlgorithmsForConnectivity}. A complete
description of the problem is in the mentioned article, there we will only
recall some basic principles (most of this section is a simplified citation from
the \cite{PolylogarithmicAlgorithmsForConnectivity}).

\noindent
{\bf Theorem:} Problem of 2-edge connectivity on dynamically updated graph with
$N$ edges could be solved in amortized time $\O(\log^4 N)$ per one operation
(addition or deletion of an edge and test if given vertices are 2-edge
connected).

\subsection{Basic principle}

We will maintain a spanning forest $F$ of graph $G$ and we will say, that tree
edge $e$ is {\I covered} by nontree edge $v,w$ if $e \in v\dots w$ (if there is
a tree path from $v$ to $w$ and $e$ is on this path).

Frederickson showed that two vertices $x$ and $y$ are 2-edge connected if and
only if they are connected in $F$ and all edges in path $x\dots y$ are covered.

So we will maintain a spanning forest $F$ together with a set $C$ of covering
edges for each non-bridge edge in $F$. Two vertices are 2-edge connected in $G$
if and only if they are 2-edge connected in $F\cup C$.

During updates, when we will delete an edge from $F$ we will need to find an
replacement in $C$. And for every deleted edge from $C$ we may need to add
several replacement edges into $C$. By carefully choosing replacement
edges we will be able to amortize the time to meet $\O(\log^4 N)$ per operation.

\subsection{Brief overview of details}

We do not discuss all the details here, they are described in \cite{PolylogarithmicAlgorithmsForConnectivity}. We only briefly overview some basic
principles to make the big picture. For details see the mentioned article.

Algorithm associates with every tree edge a nontree edge that covers it. When we
add $u,v$ edge and $u$ and $v$ are already connected in the $F$ we should add
$u,v$ edge as cover edge for all tree edges on the $u\dots v$ path. It would be
slow to directly update all the affected edges so we will distribute this
information with lazy updates through the top tree clusters -- each cluster has
its own cover information which it distributes to its path children when the
cluster is splitted.

Deletion of an uncovered tree edge is quite easy, we just \Cut{} it from the
$F$. Otherwise we need to find some replacement edges for edges that stops to be
covered. To accomplish easy finding of replacement edges the algorithm
associates with every nontree edge a level (number $< \log N$) -- each nontree
edge starts with level 0 and levels may be only increased.

For each $i$ we will define a graph $G_i$ as graph induced by only edges of
level at least $i$ together with the $F$. For each tree edge $e \in F$ we will
maintain a cover level as maximal level of nontree edge covering it (thus
maximal level $i$ for which $e$ is in a 2-edge connected component of $G_i$).
When some vertices are 2-edge connected on level $i$ they 2-edge connected on
all levels $\le i$.

When updating the cover information after deleting some edge we use these levels
as guide for finding new cover edges. Major part of the time complexity of the
whole algorithm is the time needed for updating data structures with nontree
egdes incident to some vertex on some level. These data structures are used for
finding replacement edges. This part is described in detail in the \cite{PolylogarithmicAlgorithmsForConnectivity}.

\subsection{Operations}
User could control the algorithm with three simple operations:

\begin{itemize}
\item {\bf Inserting an $u,v$ edge} -- When given vertices are not connected in
the $F$ just connect them by \Link, otherwise add a nontree edge and do cover on
the path cluster for path $u\dots v$.

\item {\bf Deleting an $u,v$ edge} -- When the $u,v$ edge is a tree edge and it is
a~bridge (not covered by any other edge) just \Cut{} it. If it is covered tree
edge, swap it with its cover nontree edge and delete it like other nontree edges.
Nontree edge is deleted firstly by {\it uncovering} $u\dots v$ path (removing all
cover informations that it may generate on the $u\dots v$ path) and then by
{\it recovering} this path (because by {\it uncovering} we may remove too much
cover information).

\item {\bf Query} if $u$ and $v$ are 2-edge connected -- This is an easy
operation when we just need to \Expose{} $u\dots v$ path and read the cover
information from the returned root cluster.
\end{itemize}

\subsection{Speed up query by disabling expensive updates}

This idea came from the original article \cite{PolylogarithmicAlgorithmsForConnectivity}
where they analyzed time complexity and postpones some enhancements.

Covering and uncovering on graph with $N$ edges takes $\O(\log^2 N)$ and
therefore one call to the \Join{} takes $\O(\log^2 N)$ and \Link, \Cut{} and
\Expose{} operations takes $\O(\log^3 N)$. Recover operations takes $\O(\log^3
N)$ for every increase of and edge level by one and because edge level is
increased at most $\O(\log N)$ times, we spent at most $\O(\log^4 N)$ time for
each edge between its insertion and deletion.

The query in this original algorithm takes $\O(\log^3 N)$, but it could be speed up
if some conditions are met. During query we don't need any updated incident
informations, we need only cover informations and these informations could be
joined in $\O(1)$ time.

If the top trees implementation could preserve the original shape of the
clusters and restore to this original form after \Expose{}, we may turn off
expensive updates in the \Join{} operation during the query and complete the
query in $\O(\log N)$ time.

This condition is met for the second implementation using topology trees, but we
cannot disable the updates in the first implementation, because the shape of the
top tree is different before and after the \Expose{} and we would lost some
incident information.

%%%%%%%%%%%%%%%%%%%%%%%%%%%%%%%%%%%%%%%%%%%%%%%%%%%%%%%%%%%%%%%%%%%%%%%%%%%%%%%%

\section{Vertex 2-connectivity}

Vertex 2-connectivity problem is only a more complicated version of an edge
2-connectivity problem discussed in the previous subsection.

Solution using top trees was introduced by Holm, Lichtenberg and Thorup 2001 in
\cite{PolylogarithmicAlgorithmsForConnectivity} and this solution works in time
$\O(\log^5 N)$ per operation. For more details see the mentioned article.
