\chapter{Examples of problems and user functions}
\label{chap:Problems}

In this section there is a list of several problems which could be solved by
Top Trees with small time complexity.

\section{Finding distance between two vertices}

This problem was originally solved in {\I A Data Structure for Dynamic Trees}
\cite{DSforDynamicTrees} by Sleator and Tarjan in 1983 and then it was adapted
for Top Trees by Alstrup, Holm, Lichtenberg and Thorup in \cite{TopTrees}.

{\bf Theorem:} Lets have dynamic collection of weighted trees with link and cut
operations. We could find length of the path between any two vertices (or find
that they are not connected) in $\O(\log N)$ time.

\medskip\noindent
{\bf Proof:} We will maintaing length of the cluster path in every cluster.

\begin{itemize}

\item $\Create$ creates cluster length equivalent to the length of the
underlying edge.

\item $\Join$ of clusters $C_1$ and $C_2$ into $C$ depends on the type of the $C$:
	\begin{itemize}
	\item If $C$ is a compress cluster of $C_1$ and $C_2$:
	Set length of the $C$'s path as sum of the lengths of $C_1$ and $C_2$.
	\item If $C$ is a rake cluster and $C_1$ is raked onto $C_2$:
	Set length of the $C$'s path as length of the $C_2$'s path (and vice
	versa if $C_2$ is raked onto $C_1$).
	\end{itemize}

\item $\Split$ and $\Destroy$ does nothing.

\end{itemize}

After that we could easily get length of the $(u,v)$-path by calling
$\Expose(u,v)$ and reading length from this cluster. Because operations
in user functions are in constant time and \Expose{} takes $\O(\log N)$
operations, we could answer on any such question in $\O(\log N)$ time.

%%%%%%%%%%%%%%%%%%%%%%%%%%%%%%%%%%%%%%%%%%%%%%%%%%%%%%%%%%%%%%%%%%%%%%%%%%%%%%%%

\section{Maximum edge weight between given vertices with interval update}
\label{sec:maximum_edge_weight}

Similarly to the previous problem this problem was originally solved in
\cite{DSforDynamicTrees} and then it was adapted for Top Trees  in
\cite{TopTrees}.

{\bf Theorem:} Let's have dynamic collection of weighted trees with operation of linking,
cutting, updating edge weight and updating edge weight on given path. We could
find maximum edge weight between any two vertices (or find that they are not
connected) in $\O(\log N)$ time.

\medskip\noindent
{\bf Proof:} We can maintain $w_{max}$ in each cluster as maximum weight on this
cluster's path and $w_{extra}$ as weight added to each edge on this path.

\begin{itemize}

\item $\Create$ creates cluster with $w_{max}=w(e)$ where $e$ is the edge for
which is this cluster created. There is no extra weight yet, so $w_{extra}=0$.

\item $\Join$ of clusters $C_1$ and $C_2$ into $C$ depends on the type of the $C$:
	\begin{itemize}
	\item If $C$ is a compress cluster of $C_1$ and $C_2$:
	Set $w_{max}$ as maximum of $w_{max}$ from clusters $C_1$ and $C_2$.
	There is no extra weight, so $w_{extra}=0$.
	\item If $C$ is a rake cluster and $C_1$ is raked onto $C_2$:
	Copy $w_{max}$ from the $C_2$. There is no extra weigh, so $w_{extra}=0$.
	\end{itemize}

\item $\Split$ have to distribute $w_{extra}$ to the children. For $C_i$ children
of splitted $C$ will operations depends on the type of the $C$:
	\begin{itemize}
	\item If $C$ is a compress cluster of $C_1$ and $C_2$:
		\begin{itemize}[$\circ$]
		\item $w_{extra}(C_i)=w_{extra}(C_i)+w_{extra}(C)$
		\item $w_{max}(C_i)=w_{max}(C_i)+w_{extra}(C)$
		\end{itemize}
	\item If $C$ is a rake cluster and $C_1$ is raked onto $C_2$:
	Apply above operation only for $C_2$ (and vice versa only for $C_1$ if $C_2$
	is raked onto $C_1$).
	\end{itemize}


\item $\Destroy$ sets weight of the underlying edge: $w(e)=w_{max}$.

\end{itemize}

Then we could just call $\Expose(u,v)$ and read $w_{max}$ or add to $w_{extra}$
of the root cluster representing the $(u,v)$-path. Everything in time complexity
of an {\sc Expose} operation, which is $\O(\log N)$.

%%%%%%%%%%%%%%%%%%%%%%%%%%%%%%%%%%%%%%%%%%%%%%%%%%%%%%%%%%%%%%%%%%%%%%%%%%%%%%%%

\section{Diameter and center of the trees}

%%%%%%%%%%%%%%%%%%%%%%%%%%%%%%%%%%%%%%%%%%%%%%%%%%%%%%%%%%%%%%%%%%%%%%%%%%%%%%%%

\section{Edge 2-connectivity}
\label{sec:edge_2_connectivity}

In time $\O(\log^4 N)$ by Holm, Lichtenberg and Thorup 2001 in \cite{PolylogarithmicAlgorithmsForConnectivity}

\todo{Do some study.}

%%%%%%%%%%%%%%%%%%%%%%%%%%%%%%%%%%%%%%%%%%%%%%%%%%%%%%%%%%%%%%%%%%%%%%%%%%%%%%%%

\section{Vertex 2-connectivity}

In time $\O(\log^5 N)$ by Holm, Lichtenberg and Thorup 2001 in \cite{PolylogarithmicAlgorithmsForConnectivity}

\todo{Do some study.}
