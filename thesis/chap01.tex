\chapter{Top Trees}

Top Trees is data structure intended to maintain informations of underlying
dynamically updated forest. They were introduced by Alstrup et al. in 2003 \cite{TopTrees}.

\section{Definition}

{\I Top Trees structure} is collection of individual {\I top trees}, each of them
represents one tree from the underlying forest. These top trees are dynamically
updates when there is some change (cut, join, update of the information in some
edge or vertex) in the underlying forest. These updates

Each {\I top tree \TT} is based on structure of a rooted binary tree -- most of
the inner vertices has two children, but in some cases there may be useful to
allow up to four children of some vertices (we will show later).

Vertices of the top tree are called {\I clusters} and each of them represents
some contraction of the underlying tree and holds informations about the
contracted part of the underlying tree.

Top Trees are capable of holding informations stored on edges of the underlying
trees and to summarize them in clusters. They are capable of updating structure
and answer queries in (amortized\footnote{If the time is amortized or no
depends on an implementation}) time $\O(\log N)$.

A special feature of the Top Trees is that users of them only defines format of
the data stored in clusters and four basic operations {\sc Create}, {\sc
Destroy}, {\sc Join} and {\sc Split} use to manipulate with clusters data. Then
user control the Top Trees by operations $\hbox{\sc Cut}(u,v)$,
$\hbox{\sc Link}(u,v)$ and $\hbox{\sc Expose}(u,v)$. Last of them brings cluster
representing the path between vertices $u$ and $v$ to the root, because the root
cluster is only cluster of the top tree, which could be accessed by user.

\section{Clusters}

There are three types of clusters in the top tree. As has been said, every of
them represents (and hold informations) about some part of the underlying tree.
Each cluster have two {\I boundary vertices}. Types of the clusters are:

\begin{itemize}

\item {\bf Base cluster} -- represents one edge of the underlying tree (and each
edge of the underlying tree has exactly one base cluster, it is 1:1 mapping),
boundary vertices are endpoints of the edge.

This cluster could appear only as leaf of the top tree.

\item {\bf Rake cluster} -- represents one way how to contract two clusters
with one common boundary vertex. Lets have two clusters $C_1(u,v)$ and
$C_2(v,w)$ next to each other around common boundary vertex $v$ and lets the
$C_1$ be the left one of them (in some topological order given for example by
indexes of the edges).

Then we construct {\I rake cluster} by {\I raking} the left cluster ($C_1$) on
the right one ($C_2$). The resulting cluster would have the same boundary
vertices as the cluster $C_2$.

\todo{Image of rake cluster}

\item{\bf Compress cluster} -- represents other contraction of the two clusters
with one common boundary vertex into one by attaching first cluster after the
other. If there are other clusters attached to the same common boundary vertex
they must be firstly {\I raked} onto one of the compressed clusters (in other words
before constructing a compress cluster the common boundary vertex must have
degree of exactly two).

If boundary vertices of the cluster $C_1$ were $(u,v)$ and boundary vertices
of the cluster $C_2$ were $(v,w)$, the cluster $C=compress(C_1,C_2)$ would have
boundary vertices $(u,w)$.

\todo{Image of compress cluster}

\end{itemize}

\subsection{Clusters model}

Compress and rake clusters has each of them two children, base clusters are
childless. Each cluster represent some part of the underlying tree and by
combination of them we could represent whole underlying tree as one cluster
(one top tree).

Compress clusters are used to represent paths in the underlying tree -- each
path could be compressed into one {\I compress tree} consisting only of compress
clusters. If there are branched separating from this path, they are recursively
represented as single clusters and then they are raked onto clusters in the path.

Underlying trees could have (and usually have) many different divisions into
paths and so the underlying trees have many different representations. Crucial
part of the top trees structure is to maintain this representation in some nice
form during updates.

\subsection{Extended clusters model}

Tarjan and Werneck in \cite{SelfAdjustingTT} suggested that in some cases it may
be useful to modify structure of the clusters and they introduced
{\I foster children} for {\I compress clusters}. In their suggestion a compress
cluster could have up to four descendants -- two normal children and up to two
foster children.

Normal children of a compress cluster are clusters from the compressed path and
foster children are clusters originating from the separating branches. In normal
cluster model they would be raked onto clusters from path and the path would be
compression of these rake clusters.

In this extended model the clusters originating from the separating branches are
firstly combined in so called {\I rake trees} -- they are maximally two rake
trees around each path vertex, one of them is raked from branches on one side of
the path and the second one is raked from branches on the other side of the
path. And these rake trees are connected as left and right foster child of the
compress cluster constructed from this part of the path.

\todo{Image of rake trees around path}

This extended model is used in the first implementation.

\section{Top trees operation}

\subsection{Expose}

\subsection{Cut}

\subsection{Link}

