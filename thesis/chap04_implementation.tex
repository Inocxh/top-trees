\chapter{Implementation and usage}
\label{chap:Implementation}

As has beeen mentioned in the Introduction both implementation are written in
\Cpp. More precisely they are written in \Cpp14 with frequent use of smart
pointers introduced in \Cpp11 and enhanced in \Cpp14, which helped a lot with
memory handling.

Source code of both implementations is attached to this thesis including
Makefile for easier compiling and testing. Source code is also published on
Github, which may be more pleasant way to explore it or use it in other
projects:

\bigskip
\centerline{\url{https://github.com/setnicka/top-trees}}
\bigskip

\section{Interface of the Top Trees structure}

Both implementations share the same interface which makes them easily
interchangeable.

Firstly user need to include \texttt{TopTreesInterface.hpp} and define classes
for holding data in vertices, edges and clusters. These classes must inherit
from generic classes for edge data, vertex data and cluster data defined in
the \texttt{TopTrees} namespace.

When using \texttt{DEBUG} options (see below) classes for edges and vertices
must implement \texttt{std::ostream\& ToString(std::ostream\&)} method too (it
is used for debug printing):

\begin{figure}[H]
\begin{verbatim}
class MyEdgeData: public TopTree::EdgeData {
public:
    int weight;
    std::string label;
    virtual std::ostream& ToString(std::ostream& o) const {
        return o << label;
    }
};

class MyVertexData: public TopTree::VertexData {
public:
    std::string label;
    virtual std::ostream& ToString(std::ostream& o) const {
        return o << label;
    }
};

struct MyClusterData: public TopTree::ClusterData {
public:
    int max_weight;
};

std::shared_ptr<TopTree::ClusterData> TopTree::InitClusterData() {
	return std::make_shared<MyClusterData>();
}
\end{verbatim}
\caption{Example of classes for edge, vertex and cluster data}
\end{figure}

\subsection{User functions}

After defining data structures user must provide definition of user functions.
User directly defines methods in the \texttt{TopTrees} namespace and without
these definitions program cannot be even compiled.

Both \Join{} and \Split{} user functions takes shared pointers to three clusters
(generic interface \texttt{TopTree::ICluster}), tow for child and one for parent.
\Create{} and \Destroy{} functions takes shared pointer to the created/destroyed
cluster and shared pointer to the underlying edge's data (data is passed as
generic type \texttt{TopTree:.EdgeData} and it have to be casted to the our data
class which we defined above).

Example of user functions is there:


\section{Testing and Graphviz output}

Both implementations have debugging output showing their current state
in Graphviz format.

Graphviz is open source program which is able to visualize several types of
graphs (tree structures included). Both implementations are able to print
Graphviz script at their standard output and this script could be translated by
Graphviz program to a PDF image.

Currently user is able to visualize state of the structure after each operation
(\Cut, \Link or \Expose) using ... \todo{Graphviz methods}
