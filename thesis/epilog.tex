\chapter*{Conclusion}
\addcontentsline{toc}{chapter}{Conclusion}

There are two main products of this thesis: Implementations of Top trees
structure and results of experiments.

Both implementations were written from scratch in \Cpp{} and they are publicly
available for any usage -- as part of this thesis or as repository on Github
(mentioned in the \cref{chap:Implementation}). They both shares the same generic
interface, which allows to easily interchange them, so each user of them can
choose the one which better fits to his needs.

Also some other works may build upon this codebase and expand it (code is shared
under public licence). Both implementations were written with efficiency in
mind, but there are still places, where they could be tweaked for even better
performance. I will be pleased if there will be active users of this piece of
code.

\bigskip

Second product of this thesis are results of experiments discussed in the last
chapter. They showed that in most cases the implementation which uses self
adjusting trees is better due to larger multiplicative constant of the
implementation with topology trees.

Only when \Join{} or \Split{} functions are time consuming and some updates
could be turned of during \Expose{} operation the second implementation could
be considered as interesting.

The experiment with edge 2-connectivity showed, that when \Expose{} operations
are significant but not the only part of all operations, the implementation with
topology trees is still slower than implementation with self adjusting trees.
But if we want to ensure quick \Expose{} operations or number of \Expose{}
operations is asymptotically larger than number of other operations, we may
use the second implementation with topology trees and with expensive
updates turned off during \Expose{}. In these circumstances it could works much
quicker for \Expose{} operation.

These results may help users of provided implementations to choose the right
implementation to fit their needs. Also other researchers in the field of Top
Trees may build upon these results.

\bigskip

And finally, important result of this thesis is a new knowledge which I earned
during working on it. It was an interesting journey through this complex data
structure, it teach me a lot of things about dynamic data structures. I hope,
that readers of this thesis would be at least half as much pleased as I am.
