\chapter{Implementation of Top Trees using Topology Trees}
\label{chap:ImplementationTopology}

This implementation is based on the article {\I Maintaining Information in
Fully-Dynamic Trees with Top Trees} \cite{TopTrees} by Alstrup, Holm,
Lichtenberg and Thorup. It builds Top trees structure on the base of topology
trees introduced in the \Cref{chap:TopologyTrees}. The update process of
topology trees and some basic overview of top clusters mapping was discussed in
that chapter, there we will introduce some details of joining, splitting and the
\Expose{} operation (\Link{} and \Cut{} operations have been described in the
mentioned chapter).

\section{Joins and Splits}

$\Join$ and \Split{} (and \Create{} and \Destroy{} for base clusters) are user
defined functions that are called on top clusters with defined endpoints. We
cannot call them directly on topology clusters, so we need to split topology
clusters to corresponding top clusters like we showed in the
\Cref{chap:TopologyTrees} and call user defined function on these top clusters.

We will remind that each topology cluster could have up to three associated
top clusters called {\I edge cluster}, {\I combined edge cluster} (edge cluster
plus one child) and the overall {\I top cluster} (combined edge cluster plus
the second child).

The external view of the topology clusters during splitting and joining is
similar as in the first implementation -- when operating on some topology
cluster we have to firstly ensure that this cluster and all of its parents are
splitted. Splitting is done recursively in the top-down manner and all splitted
clusters have to be joined after completing all operations in the bottom-up
manner. This is done by logging all splitted topology clusters into some list
and joining all of them after completing current operation.

\subsection{Joining}

We will recall that if the topology cluster $C$ is not base cluster on the
lowest level it could have one or two children (mark them as $C_1$ and $C_2$).
When joining the $C$ we have to do these operations:
\begin{enumerate}

\item If $C$ has only one child $\rightarrow$ Just copy $C_1$'s {\I top cluster}
(with endpoints) into the $C$'s {\I top cluster} and end.

\item If there is a normal edge between $C_1$ and $C_2$ $\rightarrow$ \Create{}
new {\I edge cluster} from this edge (otherwise initialize dummy one).

\item If the $C_1$ is a topology top cluster:
\begin{itemize}
\item If we created an {\I edge cluster} $\rightarrow$ \Join{} $C_1$'s {\I top
cluster} with the {\I edge cluster} into {\I combined egde cluster} (depending
on $C_1$'s shape set new cluster's endpoints as rake or compress cluster).
\item Otherwise copy $C_1$'s {\I top cluster} into {\I combined edge cluster}
(with updating endpoints).
\end{itemize}
Otherwise just copy {\I edge cluster} into {\I combined edge cluster}.

\item If the $C_2$ is a topology top cluster:
\begin{itemize}
\item If there is valid {\I combined edge cluster} (if $C_1$ is a topology top
cluster or there is a normal edge) $\rightarrow$ \Join{} $C_2$'s {\I top cluster}
with the {\I combined edge cluster} into $C$'s {\I top cluster} (depending
on $C_2$'s shape set new cluster's endpoints as rake or compress cluster).
\item Otherwise copy $C_2$'s {\I top cluster} into $C$'s {\I top cluster}.
\end{itemize}
Otherwise just copy {\I combined edge cluster} into $C$'s {\I top cluster}.

\end{enumerate}

We have done at most two calls to the \Join{} user function and one to the
\Create{} user function.

\subsection{Splitting}

Split procedure is opposite to the join procedure. Endpoints of all top clusters
are correctly set by the join procedure so we have to only do \Split{} and
\Destroy{} operations in the opposite way.

If there is a normal edge in the topology cluster we firstly \Split{} $C$'s
{\I top cluster} into {\I combined edge cluster} and $C_2$'s {\I top cluster}
and then \Split{} {\I combined edge cluster} into {\I edge cluster} and $C_1$'s
{\I top cluster}. Finally just \Destroy{} the {\I edge cluster}.

When there is a subvertice edge just Split $C$'s {\I top cluster} into children
{\I top clusters}. If some of the children is not a topology top cluster we just
do copy instead of \Split{} (like in the join procedure).

We have done at most two calls to the \Split{} user function and one to the
\Create{} user function.


\section{Expose}
