\chapter{Implementation of Top Trees using Topology Trees}
\label{chap:ImplementationTopology}

\section{Joins and Splits}

Like in the first implementation when operating on some cluster (during the
update procedure) we have to firstly ensure that this cluster is splitted.
Splitting is done recursively in the top-down manner and all splitted clusters
have to be joined after completing all operations in the bottom-up manner.

This is done by logging all splitted topology clusters into some list and
joining all of them after completing current operation.

\subsection{Join}

We will recall that if the topology cluster $C$ is not base cluster on the
lowest level it could have one or two children (mark them as $C_1$ and $C_2$).
Also we will remind that each topology cluster could have up to three associated
top clusters (called {\I edge cluster}, {\I combined edge cluster} and the
overall {\I top cluster}), for more details see the \Cref{chap:TopologyTrees}.

When joining the $C$ we have to do these operations:
\begin{enumerate}

\item If $C$ has only one child $\rightarrow$ Just copy $C_1$'s {\I top cluster}
(with endpoints) into the $C$'s {\I top cluster} and end.

\item If there is a normal edge between $C_1$ and $C_2$ $\rightarrow$ Create new
{\I edge cluster} from this edge (otherwise initialize dummy one).

\item If the $C_1$ is a topology top cluster:
\begin{itemize}
\item If we created an {\I edge cluster} $\rightarrow$ Join $C_1$'s {\I top
cluster} with the {\I edge cluster} into {\I combined egde cluster} (depending
on $C_1$'s shape set new cluster's endpoints as rake or compress cluster).
\item Otherwise copy $C_1$'s {\I top cluster} into {\I combined edge cluster}
(with updating endpoints).
\end{itemize}
Otherwise just copy {\I edge cluster} into {\I combined edge cluster}.

\item If the $C_2$ is a topology top cluster:
\begin{itemize}
\item If there is valid {\I combined edge cluster} (if $C_1$ is a topology top
cluster or there is a normal edge) $\rightarrow$ Join $C_2$'s {\I top cluster}
with the {\I combined edge cluster} into $C$'s {\I top cluster} (depending
on $C_2$'s shape set new cluster's endpoints as rake or compress cluster).
\item Otherwise copy $C_2$'s {\I top cluster} into $C$'s {\I top cluster}.
\end{itemize}
Otherwise just copy {\I combined edge cluster} into $C$'s {\I top cluster}.

\end{enumerate}

We have done at most two calls to the Join user function and one to the Create
user function.

\subsection{Split}

Split operation is opposite to the Join operation. Endpoints of all top clusters
are correctly set by the Join operation so we have to only do Split and Destroy
operations in the opposite way.

If there is a normal edge in the topology cluster we firstly Split $C$'s
{\I top cluster} into {\I combined edge cluster} and $C_2$'s {\I top cluster}
and then Split {\I combined edge cluster} into {\I edge cluster} and $C_1$'s
{\I top cluster}. Finally just Destroy the {\I edge cluster}.

When there was a subvertice edge just Split $C$'s {\I top cluster} into children
{\I top clusters}.

If some of the children is not a topology top cluster just do copying instead of
splitting (like in the Join procedure). We have done at most two calls to the
Split user function and one to the Create user function.
