\chapter{Implementation of Top Trees using self adjusting trees}
\label{chap:ImplementationSelfAdjusting}

This implementation is based on article {\I Self-Adjusting Top Trees}
\cite{SelfAdjustingTT} by Tarjan and Werneck and uses the extended clusters
model with foster children discussed in the \Cref{chap:TopTrees}.

\section{Construction}

Tarjan and Werneck in \cite{SelfAdjustingTT} suggested this construction:
\begin{enumerate}
\item Choose root $r$ as a vertex with degree one.
\item Orient all edges in the tree containing vertex $r$ towards the vertex $r$.
\item Divide tree into paths starting in some leaf and continuing along the
direction of the edges -- the first path will end in the root $r$ and became the
{\I root path}, other paths end up being connected to some existing path.
\item Recursively compute clusters to represent each path incident to the
root path and create {\I rake trees} from these incident paths.
\item Create binary tree of compress clusters to represent the root path and
connect rake trees as foster children.
\item If there are some unused vertices of degree one start the process again
from any of these vertices to construct another top tree.
\end{enumerate}

In my implementation I choose equivalent construction but in more recursive
manner. I started the same way by choosing the root $r$ as vertex with degree
one, but I don't divide the tree into paths.

Starting from the second vertex we choose one neighbour as continuation of the
path and recursively called the same function on all other neighbours. Recursion
returns clusters representing each of the subtrees and then they could be
raked into left and right rake trees and saved into this vertex for future use.

When compressing the path into compress clusters we just look into the common
vertex of compressed clusters and if there are saved rake trees we connect them
as left and right foster children.

This construction is easier to implement and gives us ability to better control
the shape of the resulting top tree. By choosing neighbors instead of directing
paths from leafs we could prefer longer paths by choosing neighbors with deepest
subtree (we firstly run \gls{dfs} to obtain depths). Longer paths are better
contracted in binary tree structure of compress clusters to obtain lower top
tree.

\section{Expose}

Expose in this implementation is based on splaying and splicing whose are used
to bring handles of given vertices to the top of theirs top trees.

Implementation of the \Expose{} operation has two parts: soft and hard expose. Soft
expose is used internally by other operations, hard expose is used only in the
\Expose{} itself.

Before moving forward we will recall some internal structure of top trees with
extended clusters model.

%In the second
%implementation with topology trees the expose is implemented by series of cuts
%and joins and this is described in the \Cref{chap:TopologyTrees} of this thesis. We
%describe here the soft and hard expose from the first implementation as they
%operate directly with the top trees clusters.

\subsubsection{Compress and rake trees}

Because of extended clusters model each top tree consist from independent
{\I compress trees} (only compress clusters as internal nodes) and
{\I rake trees} (only rake clusters as internal nodes).

Whole top tree is one compress tree (which represents the {\I root path}). It
has base clusters as leafs and root of rake trees as foster children (these
foster children are other paths connected to the root path). These rake trees
have rake clusters as theirs internal nodes and base or compress clusters as
theirs leaf. And so on.

This division of the top tree into smaller blocks could be used to expose given
pair of vertices in the root of the top tree. We will use operations of
{\I splay} and {\I splice} introduced by Sleator and Tarjan in
\cite{SelfAdjustingBST}.

\subsubsection{Split and Join operations}

Before doing any operation that changes shape of a top tree all nodes involved
in this operation must be splitted (and all their parents on the way to the root
of this tree). This is crucial because after changing shape of the top tree
a data stored in these nodes may be changed (for example depth of subtree bellow
this node).

Split operations have to be done in top-down manner (starting from the root). The
easiest way how to accomplish this is to have flag in each node if it is
splitted and recursively split parent before splitting current node. All
splitted nodes should be logged into some list to easily join all of them after
completing current operation.

Joining is done in opposite direction, in bottom-up manner (ending in the root).
We will assume that before doing anything with any node during splaying and
splicing operations we firstly split this node and after completing the entire
expose operation we will call join on all splitted nodes.

\subsection{Splaying}

Splaying is originally a heuristic for balancing binary trees which uses an idea
that often used nodes should be near the root of the tree. Each operation (find,
delete, \dots) on a vertex in splay tree is preceded by splaying this vertex
which moves this vertex to the root of the tree.

Splaying is done by rotations or double rotations (which are called {\I zig-zig}
or {\I zig-zag} rotations). That moves target vertex up by one or two levels
preserving all vertices in the right order.

Although some not so often used vertices may be in $\O(n)$ distance from the
root, Sleator and Tarjan in \cite{SelfAdjustingBST} proved that all operations
work in amortized time $\O(\log n)$ per operation.

In the Top Trees structure We will use {\I guarded splays} that works exactly
the same way as normal splays, but it stops splaying when reaching a guard (some
node). Normal splay has as guard root of the whole tree, but we want to do
splays limited only inside compress or rake tree (not to mix compress and
rake clusters).

Implementation of the splay is straightforward. Only noticeable detail (which
Tarjan and Werneck mentioned in the \cite{SelfAdjustingTT}) is that foster
children are not affected by any rotations, they always keep the same parents.

\subsection{Splicing}

Only by splaying we are not able to expose a path containing both given
vertices. And we are not even able to carry compress clusters over rake trees
(because rake trees are connected as foster children under compress trees). For
that we need to change partitioning of the top tree into paths.

Lets take some vertex $v$ which is internal to some path $a,b$ (so there is
compress cluster around this vertex) and which has node representing path $c,v$
in its connected rake tree. We may change partitioning into paths by removing
one half of the original path (for example $v,b$), connect it into rake tree,
removing node representing path $c,v$ from the rake tree and changing the
compress cluster that it now represents path $a,c$.

\begin{figure}[H]
\centering
\asyinclude{pic/chap05_splice.asy}
\caption[Splice around vertex $v$]
{Splice around vertex $v$ and changing $v,b$ for $v,c$ (grayed clusters are rake
clusters).}
\end{figure}

Splicing is used after splaying and its task is to move cluster, which is leaf
of some rake tree, to the compress tree above it.

Tarjan and Werneck in the \cite{SelfAdjustingTT} only described the main idea in
the general case, but implementation details had to be worked out. I decided
to do {\I left splice} (replacing left child of the compress cluster) which was
original choose of Tarjan and Werneck, but the procedure would work the same
way if we choose to replace the right child.

My implementation does splicing in this way:
\begin{enumerate}
\item Prepare empty lists of left and right neighbors.
\item Starting from given node $N$ go up until reaching node $N_c$ in compress
tree. During that add left and right neighbors into neighbors lists and destroy
old internal nodes of these rake trees.
\item Add left child of the $N_c$ to the appropriate list of neighbors and
connect $N$ as the new left child of the $N_c$ (that makes $N$ part of the above
compress tree).
\item Construct new left and right foster children (rake trees) from left and
right neighbors lists respecting their order. New internal rake nodes have
to be constructed (and added into list of nodes for joining).
\end{enumerate}

To assure that we do not broke connectivity of $N_c$ with above clusters by
replacing left child we need to check children of the $N_c$. If the above
cluster (parent) is compress cluster, one of boundary vertices is common vertex
of the parent. And if the parent is rake cluster, one of boundary vertices is
the vertex around which is the parent raked. If this connection is in the left
child we need to flip children (otherwise child with the connection would be
moved into rake tree and connection would broke).

This detail was not discussed by Tarjan and Werneck -- they only suggested
transforming top trees into some normalized form which would work for both
splaying and splicing. But they did not mention some corner cases where such
checking and flipping is needed.

\subsection{Soft expose}

Soft expose is the first part of the \Expose{} operation. When it is called as
{\I soft\_expose(u,v)} its target is to bring path $u\dots v$ as subpath into
the root cluster of corresponding top tree (so after soft expose there should
be some path $a\dots u\dots v\dots b$ in the root cluster). Truncating the root
cluster to contain only $u\dots v$ path is quest for the hard expose operation.

Soft expose takes handles of both vertices and brings them to the top of their
top trees. If both vertices are in different components (they are not connected
by a path) both the handles of $u$ and $v$ are brought to the roots of
corresponding top trees using series of local changes in the top trees
(similarly if $u=v$).

When they are in the same component (they are connected by a path) firstly the
handle of $u$ is brought to the root of corresponding top tree. If the current
root cluster is also handle of $v$ we are done, otherwise the handle of $v$ is
brought as close to the root as possible (but not replacing the handle of $u$ as
root).

\subsubsection{Algorithm}

Exact procedure for one vertex (as described by Tarjan and Werneck in
\cite{SelfAdjustingTT}) is following:

\begin{enumerate}
\item Local Splays -- Starting from the handle of given vertex:
  \begin{enumerate}
  \item Splay current node inside compress tree (that makes it root of that compress tree)
  \item If reaching a root cluster (there is no parent) $\rightarrow$ stop the cycle.
  \item Splay parent inside rake tree (parent of a root of a compress tree is always rake cluster).
  \item Take parent of that node (it is compress cluster having this rake tree as foster child)
  and repeat the cycle.
  \end{enumerate}
\item Splices -- Starting from the handle of given vertex splice current vertex and
moving to its parent. That in every step moves handle of the given vertex across
one rake tree to the above compress tree. Finally it moves this handle into the
root compress tree.
\item Global Splay -- Perform splay on the handle of given vertex (now in the
root compress tree) to make it root of this top tree.
\end{enumerate}

Firstly we expose $handle(u)$ using above procedure. After that we expose
$handle(v)$ with $handle(u)$ as the guard (to assure that it remains at the
top). If both vertices (and so both handles) are in different top trees both of
them ends as roots of theirs top trees -- this situation is not interesting
anymore, so we will assume that they are in the same top tree.

Tarjan and Werneck discussed special situation when one of the vertices has
degree one (and its handle is not a compress cluster around this vertex), in
this case they realized that mentioned procedure leads to exposing handle of the
second vertex with first vertex as one of its endpoints.

Another special case, which they did not discussed, is situation when both
vertices have degree one -- in that case before starting procedure for the
second vertex we have to ensure that the base cluster of the first vertex is not
the left child of root cluster (otherwise it would be moved to rake tree when
splicing the handle of the second vertex, what we do not want).

If handles for both vertices are the same there is nothing to do (both handles
are occupied by the root cluster). When they are different the second handle
ends as one of the first handle children. To make hard expose easier we will
assume that this is the right children (otherwise we just flip left and right
children).

And where is path $u\dots v$ located? There are three possibilities (notice that
we are still operating with extended clusters model, where rake trees are
connected as foster children -- in standard top tree model there could be
intermediate rake clusters and the cluster with path $u\dots v$ could be located
deeper):

\begin{figure}[H]
\centering
\asyinclude{pic/chap05_soft_expose_cases.asy}
\caption{Three possible cases where cluster with endpoints $u,v$ could be located
after {\I soft\_expose(u,v)}.}
\end{figure}

\begin{enumerate}

\item Root cluster itself -- When both vertices have degree one, they are endpoints
of the root cluster and so their handles are represented in the root cluster.

\item Child of the root cluster -- When one vertex has degree at least two its
handle is compress cluster. In that case the root cluster represents some path
$a\dots u\dots v$ or $u\dots v\dots b$ and desired path $u\dots v$ is
represented in its child (assume that it is the right child, otherwise flip them).

Another case when path is in child of the root cluster is when $u$ and $v$ are
connected directly by an edge. In this case the path is represented by base cluster
and so it might be child of some compress cluster.

\item Grandchild of the root cluster -- When both vertices have degree at least
two both handles are represented by compress clusters. Child of the root cluster
is limited from the first side and its again compress cluster (the second handle).
The grandchild of the root cluster is limited from the second cluster and so it
represents our path (let again assume it is the right child, otherwise flip
children).

\end{enumerate}

To ensure that the root cluster represents only the path $u\dots v$ we need the
hard expose operation.

\subsection{Hard expose}

Hard expose is the second part of the \Expose{} operation, it follows the soft
expose and its job is to truncate the path in the root cluster that it contains
only the exposed subpath.

Ideal situation is when $u$ and $v$ are endpoints of the root cluster, it is
possible when both of them have degree one or when all other clusters incident
to the root cluster are raked onto the root cluster.

But in general root cluster could represent some path $x\dots y$ with path $u\dots v$
as subpath. In this case we need to temporarily convert ends of this path (paths
$x\dots u$ and $v\dots y$) into rake clusters so the compress tree would represent the
path $u\dots v$ with these ends raked onto this path.

As we discussed at the end of the soft expose operation, wanted subpath may be
located in the root cluster itself, in its (right) child or in its (rightmost)
grandchild.

Tarjan and Werneck (in \cite{SelfAdjustingTT}) came with simple trick -- temporarily
convert compress clusters above wanted cluster (at most two compress clusters as
we realized above) to rake clusters ({\I rakerizing} them). Because we are using
left rake clusters the resulting rake cluster will have the same boundaries as
its right child -- this is the reason why we needed to have cluster with the
wanted path on the right side.

After rakerizing clusters (we have to not forget to split them before the
operation and join them after it) we have the $u\dots v$ path represented in
the root cluster, expose procedure is finished.

But we brought the top tree in some corrupted form by rakerizing (at most two)
compress clusters. Prior the beginning of the next operation (Expose, Link, Cut
or Search) we have to undone this and return the top tree into its original
form, otherwise the amortization arguments would not work.

When rakerizing we save all rakerized compress clusters into some list and
before any other operation (\Cut, \Link or \Expose) we call the \Restore{} operation.
The \Restore{} operation checks this list and if there are some vertices it changes
them back to compress clusters. All that we could do in constant time at start
of every operation.

\section{Cut}

Implementing \Cut{} operation is quite easy thanks to the soft expose operation.
First step of the $\Cut(u,v)$ operation is to do $soft\_expose(u,v)$ which brings
the top tree into state described at the end of the soft expose subsection
-- depending on the degrees of $u$ and $v$ the cluster representing
$(u,v)$ edge will be the rightmost child or grandchild of the root cluster of the
corresponding top tree.

\begin{figure}[H]
\centering
\asyinclude{pic/chap05_cut.asy}
\caption[Example of cut]
{Example of cut. Cluster $(u,x)$ is the rightmost cluster from $B$ and
$B'$ is $B$ without this cluster (cluster $(v,y)$ and $D'$ in the same way).}
\end{figure}

We have to destroy the base node representing $(u,v)$ and remove connection
between these two handles. After that we have to reorganize clusters to ensure
all clusters have both children.

When one of vertices have degree only one we are removing leaf edge and there
will be only one resulting top tree (or even no resulting top tree when both
vertices have degree only one and the whole top tree consists only from this
edge). Otherwise we have to split the top tree in the middle points of the two
compress clusters (whose we bring to the root by soft expose).

Starting from the root we detach the right child of the current cluster -- but
we could not leave the cluster in this form, we need to find new right child.

If there are some nodes in left or right foster children (rake trees) we could
take one leaf from the rake tree (base or compress cluster) and make it the new
right child of the current cluster. When there is only one cluster in the rake
tree it is easy, but what to do if there are clusters?

When taking cluster from the left rake tree we want to get the leftmost one
(or the rightmost one in the right rake tree) to maintain order. We splay on
the chosen clusters parent (which is rake cluster) to make it the root of this
rake tree, or chosen cluster will be the left (right) child of this cluster and
the rest of the rake tree will be the other child. Now we can simply remove this
parent rake cluster, use our chosen cluster as the right child and rest of the
rake tree as the new left (right) foster child.

The last case is when the current compress cluster have not any foster children,
in that case we simply remove this compress cluster and replace it by its left
child.

This whole procedure is done at most twice (third level is the $u,v$ base
cluster itself). It produces (two) new root clusters clusters of the resulting
top trees after cutting the $u,v$ edge.

\section{Link}

$\Link$ is similar to the \Cut{} operation but in the opposite way. First step
during the $\Link(u,v)$ operation is bringing both $u$ and $v$ to the top in
corresponding top trees. This could be simply done by calling
$soft\_expose(u,v)$.

Special case is when we are joining solitary vertices. In that case we simply
construct new base cluster and return it as the new top tree.

Otherwise we choose one vertex (and its handle) as the root of the final top tree,
for easier construction we choose the vertex with bigger degree. Then starting
from the second cluster (handle of the second vertex) we firstly move its right
child into one of its foster children (rake trees). If there is no rake tree it
is simple, otherwise we just construct new rake cluster connecting the existing
rake tree as one of its child and the former right child of the current compress
cluster as its another child.

\begin{figure}[H]
\centering
\asyinclude{pic/chap05_link.asy}
\caption[Example of link]
{Example of link. Subtree $B'$ is $B$ with the $(u,x)$ cluster connected
as its rightmost child. Subtree $D'$ and cluster $(v,y)$ similarly.}
\end{figure}

After this move we simply connect the new $u,v$ base cluster as right child and we
are done with this (lower) compress cluster. For the root compress cluster we
will do similar procedure with the only difference -- instead of the $u,v$ base
cluster we will connect just modified compress cluster as the right child. This
leads to constructing the final top tree.
