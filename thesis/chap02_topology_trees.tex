\chapter{Topology Trees}
\label{chap:TopologyTrees}

The Topology Trees data structure introduced by Frederickson
\cite{DSforDynamicallyMaintainingRootedTrees} is used as the basic building
block in the second implementation. We devote this chapter to basic
understanding of how this data structure works and what must be done to use
it in our case.

Basic idea of the Topology Trees is to divide a tree (or in general a forest)
with maximal degree of three into recursive {\I clusters}. Trees with higher
degree has to be firstly {\I ternarized} -- their vertices have to be splitted.
We will discuss the ternarization later, let's suppose we already have
ternarized tree.

A collection of rooted binary trees (called {\I topology trees}) is built from
these recursive clusters, one for each tree in the original forest.

Cluster in topology tree is a set of maximally two nodes (vertices of the
original tree or other topology tree clusters) with edge between them (but
without edges to neighbours) and with maximally three neighbours, better
definition will conclude below. Nodes not connected by edge cannot be in common
cluster, thus clusters represents some contracted subtrees of the original tree.

Note that clusters in topology trees are slightly different than clusters in
top trees. In top trees clusters are generalized edges (with two endpoints)
representing contracted subtree between two vertices, but clusters in topology
trees are generalized vertices (with maximally three outgoing edges).

\section{Definition and properties}

\subsection{Topology clusters and clusterization}

A {\I cluster of order $k$} is a set of $k$ nodes connected by edges (there must
be a path from each to each). A {\I clusterization of order $k$} of graph is
division of this graph into clusters that:
\begin{itemize}
\item Each cluster has maximal order $k$.
\item Each vertex of the original graph is contained in exactly one cluster.
\end{itemize}

The {\I Topology clusterization} of a tree (with maximal degree 3) is clusterization
of order 2 whose clusters (called {\I topology clusters}) must meet these conditions:
\begin{enumerate}
\item {\bf Neighbours limit}: Every topology cluster has degree (number of
outgoing edges to neighbours) maximally 3.
\item {\bf Simple crossroads}: Clusters with three neighbours must have order 1 (only one vertex).
\item {\bf Minimality}: There is no cluster that could be merged with its neighbour
without violation these rules.
\end{enumerate}

Because of that there are only 8 types of valid topology clusters:

\todo{picture of valid topology clusters}

\subsection{Topology tree}

The {\I topology tree} is a rooted binary tree where each
level of it represents some topology clusterization of the original tree. On the
lowest level each of the original vertices is an independent basic cluster.
These clusters are joined in the first level of topology clusterization,
resulting clusters are contracted and acts as nodes for above level and so on.

Each inner cluster has at most two children on the level below (the clusters
from which is it contracted) and at most one parent (topology clusters without
parent are roots of topology trees). Among these tree edges each topology
cluster is connected with its neighbours on the same level -- each topology
cluster has at most three outgoing edges.

\subsection{Height of a topology tree}

Frederickson in \cite{DSforDynamicallyMaintainingRootedTrees} proved that each
level of topology clusterization has at most $5/6$ clusters of the previous
level. Number of clusters on the lowest level corresponds to the number of
vertices of the original tree (denote it as $N$) and at $k$-th level there are
at most $N\cdot(5/6)^k$ clusters. Therefore the height of the topology tree for
tree with $N$ vertices could be estimated as $\O(\log N)$.

\section{Updates -- Cut and Links}

After a change in the underlying forest (adding or removing an edge) we must
update the whole topology trees structure -- after link (adding edge) two
topology trees are joined into one and after cut (removing edge) one topology
tree is splitted into two topology trees.

This change has to be propagated on all levels of a topology tree and
corresponding topology clusters (and their neighbours) have to be updated. The
main idea is that this operation should do constant work on each level which
leads to the time $\O(\log N)$ for each operation.

Update process was originally described by Frederickson in
\cite{DSforDynamicallyMaintainingRootedTrees} but my implementation is based on
process described in Martin Mareš's master thesis \cite{DGA} (in Czech). The
process is the same but in the second work was described more clearly.

\subsection{Update process}

Both Link and Cut starts by modifying the original forest which will break some
constraints in the lowest level of our topology trees. Whole update process
goes from the lowest level up and repairs broken constraints.

Update process work in phases (one phase for level in topology tree) and uses
three lists listed below:
\begin{itemize}
\item Delete list -- clusters to be deleted, initially is empty.
\item Change list -- clusters whose are changed and needs recomputation, initially
it contains basic clusters at endpoints of the added/removed edge.
\item Abandon list -- clusters whose are (for some reason) without parent and
needs some, initially is empty.
\end{itemize}

In each phase update process processes all clusters from all three lists and
prepares lists of clusters from above level for the next phase. When there are
no clusters to process it ends.

Firstly for all clusters from delete list: We delete this cluster (and
disconnect all outer edges) and if it is the only child of its parent we add
parent to the delete list for the next phase, otherwise we add the second child
into current change list, because we need to recompute it's neighbours (and
that transitively ensures that parent would get into new change list).

Next for all clusters in change list: If they have sibling (the second child of
their parent) and they are connected with him by edge then everything is correct
-- we only update their neighbours (based on it's children neighbours) and add
the parent into next change list (we need to update parent's neighbours too).

If cluster from change list has sibling but they are not connected by edge (for
example the initial state after cut operation) their parent is no longer a
valid cluster -- therefore we add their parent into next delete list and move
both clusters into abandon list (because they are without valid parent).

And finally for all other clusters (rest of change list and abandon list) we do
this process:
\begin{itemize}
\item When cluster had no neighbours we found a new root cluster
$\rightarrow$ we just save it into list of root clusters.
\item When cluster has three neighbours and one of the neighbours is {\I single
cluster} (cluster with only one neighbour) $\rightarrow$ we join them together
under one parent, resulting cluster would have two neighbours. Depending on both
clusters parent there are several options:
	\begin{itemize}
	\item If one of the clusters has parent we reuse it and add this parent
	into next changed list.
	\item If both clusters have parent we choose one, reuse it (adding it
	into next change list) and add second one into next delete list.
	\item If both clusters were without parent we have to create a new one
	and add this new parent into next abandon list.
	\end{itemize}
\item When cluster has three neighbours but no neighbour is single cluster
$\rightarrow$ we cannot join cluster with any of its neighbours, therefore we
only ensure there is parent of this cluster (if parent exists we add it into
next change list, otherwise we create it and add it into next abandon list).
\item If cluster has degree (number of neighbours) maximaly 2 and it has
neighbour without sibling such that degree of cluster plus degree of this
neighbour is maximally 4 we join them together. The resulting parent cluster
will have degree 2 (because we encapsulate the common edge, which we count
twice, into parent cluster). Ensuring the parent cluster is similar as in the
second case. If there is no such neighbour we only ensure that parent of this cluster exists
(similar as in the third case).
\end{itemize}

This process takes $\O(1)$ per level and $\O(\log N)$ for the whole topology tree.

%%%%%%%%%%%%%%%%%%%%%%%%%%%%%%%%%%%%%%%%%%%%%%%%%%%%%%%%%%%%%%%%%%%%%%%%%%%%%%%%

\section{Ternarization of a tree}

One of the things we have to deal with when using Topology trees as an executive
layer for Top trees are degrees of vertices -- the Top Trees structure have to
work on trees of any degree but Topology trees works only on trees with max
degree 3.

We have to {\I ternarize} each vertex -- turning vertex with higher degree into
chain of multiple vertices with degree 3 -- and keep this ternarization during
cuts and joins.

Let's have a vertex of degree $D\ge4$. It could be splitted into chain of
multiple vertices with maximal degree 3 by these simple steps:
\begin{itemize}
\item Create $D - 2$ {\I subvertices} and set their {\I superior vertex} as the
original vertex.
\item Connect subvertices by edges (called {\I subvertice edges}) into one chain (first
and last vertex will have one edge used, inner will have two edges used)
\item Disconnect neighbours from the original vertex and connect them to these
subvertices (there are exactly the right number of free neighbour slots).
\end{itemize}

But the Cut and Link operations in the Top trees works with the original
vertices, we have to build some mapping on internal Topology trees cuts and link
operations.

\subsection{Ternarization during Cut operation}

Both endpoints of the cut could be processed independently. When cutting on
vertex with degree maximally 3 nothing had to be done and we may simply call
internal cut operation.

When cutting on vertex which is splitted into several subvertices the operation
is more difficult.

Firstly we need to find right subvertex which incidents with the edge. This
could be done easily by some list of pointers. After finding the right subvertex
and edge we call the internal cut on this edge, but this operation decreases
degree of this subvertex. According to situation we have to do several repair
steps:

\begin{itemize}
\item {\I If this is inner subvertex of the subvertice chain:} We remove it
from the chain, we need two internal cuts on edges to both neighbours and one
internal link to directly reconnect neighbours.
\item {\I If this is outer vertex of the subvertices chain and there is at least
one inner vertex of the chain (when degree of the original vertex was
at least 5):} We ``steal'' one outer edge from the neighbour (which is
an inner vertex of the subvertice chain) by one internal cut and one internal
link and then we continue as in the first case by removing this inner vertex.
\item {\I If this is outer vertex of the subvertice chain and there is no inner
vertex:} We have to join subvertices into the original vertex. We cut each edge
to neighbour and link it back to the original vertex, finally we cut the edge
between both subvertices and delete these subvertices.

Because degree of the original vertex is 3 after this operation we do exactly
4 cuts and 3 links.
\end{itemize}

In each case we have to do only constant number of inner cuts and links. All
these inner operations work in the $\O(\log N)$ time so the time complexity of
the whole Cut operation is $\O(\log N)$ (but the multiplicative constant could
be really large).

\subsection{Ternarization during Link operation}

As in the Cut operation both endpoints of the link could be processed
independently. We firstly need to ensure that both endpoints have degree at
most 2. When linking such vertices we not need to do anything and we just simply
call internal link operation.

When linking vertex with degree 3 or more we need to split it into subvertices
(or add a new subvertex to the existing subvertices chain).

When degree of the vertex was exactly 3 it is not splitted and we have to split
it -- as in the procedure above we create two subvertices, we link them by one
edge and then for all existing neighbours we cut them from the original vertex
and link them to one of the subvertices. One subvertex ends with degree only
2 and this subvertex we use as an endpoint for the main link operation.

Otherwise when vertex is already splitted into subvertices we have to create
a new subvertex and insert it into the subvertices chain. Easiest is to do cut
between the first and the second subvertex in the chain and then two links --
between the first and the new and between the new subvertex and the second subvertex.
Then we use the new subvertex as an endpoint for the main link operation

In each case we have to do only constant number of inner cuts and links. All
these inner operations works in the $\O(\log N)$ time so the time complexity of
the whole Link operation is $\O(\log N)$ (but as in the Cut's case the
multiplicative constant could be really large).

\section{Mapping Top trees clusters}

Usage of Topology trees as backend for Top trees was described by Alstrup, Holm,
Lichtenberg and Thorup in \cite{TopTrees}. They described the need of
ternarization and how to make operations with the topology clusters to Split and
Join operations on Top trees clusters.

Outgoing edges acts differently in topology clusters and top trees clusters --
in topology clusters outgoing edges are not parts of topology clusters, but
top trees clusters are based on these edges. Even that this is a major
difference the mapping could be done quite easily.

But before mapping we have to deal with fake subvertices and subvertice edges
added by ternarization.

\subsection{Subvertices and subvertice edges from the Top trees perspective}

During ternarization we added additional subvertices and subvertice edges
into into the graph. That is needed by topology trees but it may be problem for
the top trees operations.

Firstly how to deal with subvertices: When performing top trees join or split
and joined cluster has subvertex as it's endpoint, we pass the superior vertex
of this endpoint to the join/split user function instead of the original
endpoint. From the user's points of view all subvertices are represented by the
original superior vertex.

And how to deal with subvertice edges? Just ignore them -- there are no top
clusters associated with them and when joining topology cluster with subvertice
edge we just rake it's children ignoring the edge.

\subsection{Associated top clusters}

With each topology cluster may be associated at most three top clusters:
\begin{itemize}
\item {\I Edge cluster} -- when there is an normal edge inside the topology cluster.
\item {\I Combined edge cluster} -- joined {\I edge cluster} and cluster from
the first child (when the first child has it's own top cluster).
\item {\I Top cluster} for the whole topology cluster (joining {\I combined edge
cluster} and cluster from the second child if there is any).
\end{itemize}

If topology cluster (or recursive it's children) contains at least one normal
edge we call it {\I topology top cluster} (because it has it's associated top
cluster). Otherwise we will call it {\I empty topology cluster}.

Originally all vertices of the original tree are empty topology clusters and by
joining them we create topology top clusters.

\subsection{Joins and Splits}

Mechanism of the joining and splitting of the topology cluster is similar as the
mechanism for original Top trees but with respect to the clusters mapping.
Details of Join and Split operations are included in the \Cref{chap:ImplementationTopology}.

\section{Expose}

\todo{Implementation of Expose in topology trees}

TODO
