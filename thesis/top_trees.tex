%%% The main file. It contains definitions of basic parameters and includes all other parts.

%% Settings for single-side (simplex) printing
% Margins: left 40mm, right 25mm, top and bottom 25mm
% (but beware, LaTeX adds 1in implicitly)
\documentclass[12pt,a4paper]{report}
\setlength\textwidth{145mm}
\setlength\textheight{247mm}
\setlength\oddsidemargin{15mm}
\setlength\evensidemargin{15mm}
\setlength\topmargin{0mm}
\setlength\headsep{0mm}
\setlength\headheight{0mm}
% \openright makes the following text appear on a right-hand page
\let\openright=\clearpage

%% Settings for two-sided (duplex) printing
% \documentclass[12pt,a4paper,twoside,openright]{report}
% \setlength\textwidth{145mm}
% \setlength\textheight{247mm}
% \setlength\oddsidemargin{14.2mm}
% \setlength\evensidemargin{0mm}
% \setlength\topmargin{0mm}
% \setlength\headsep{0mm}
% \setlength\headheight{0mm}
% \let\openright=\cleardoublepage

% Glyphtounicode cause unicode text easily searchable a copyable from PDF
\input{glyphtounicode}
\pdfgentounicode=1

%% Character encoding: usually latin2, cp1250 or utf8:
\usepackage[utf8]{inputenc}

%% Further useful packages (included in most LaTeX distributions)
\usepackage{amsmath}        % extensions for typesetting of math
\usepackage{amsfonts}       % math fonts
\usepackage{amsthm}         % theorems, definitions, etc.
\usepackage{bbding}         % various symbols (squares, asterisks, scissors, ...)
\usepackage{bm}             % boldface symbols (\bm)
\usepackage{graphicx}       % embedding of pictures
\usepackage{fancyvrb}       % improved verbatim environment
\usepackage{natbib}         % citation style AUTHOR (YEAR), or AUTHOR [NUMBER]
\usepackage[nottoc]{tocbibind} % makes sure that bibliography and the lists
			    % of figures/tables are included in the table
			    % of contents
\usepackage{dcolumn}        % improved alignment of table columns
\usepackage{booktabs}       % improved horizontal lines in tables
\usepackage{paralist}       % improved enumerate and itemize
\usepackage[usenames]{xcolor}  % typesetting in color

\usepackage[footnote,acronym,nomain]{glossaries}
\setglossarystyle{list}
\makeglossaries

\usepackage{easy-todo}

%%% Basic information on the thesis

% Thesis title in English (exactly as in the formal assignment)
\def\ThesisTitle{Comparison of Top trees implementations}

% Author of the thesis
\def\ThesisAuthor{Jiří Setnička}

% Year when the thesis is submitted
\def\YearSubmitted{2017}

% Name of the department or institute, where the work was officially assigned
% (according to the Organizational Structure of MFF UK in English,
% or a full name of a department outside MFF)
\def\Department{Department of Theoretical Computer Science and Mathematical Logic}

% Is it a department (katedra), or an institute (ústav)?
\def\DeptType{Department}

% Thesis supervisor: name, surname and titles
\def\Supervisor{Mgr. Vladan Majerech, Dr.}

% Supervisor's department (again according to Organizational structure of MFF)
\def\SupervisorsDepartment{Department of Theoretical Computer Science and Mathematical Logic}

% Study programme and specialization
\def\StudyProgramme{Computer Science}
\def\StudyBranch{Discrete Models and Algorithms}

% An optional dedication: you can thank whomever you wish (your supervisor,
% consultant, a person who lent the software, etc.)
\def\Dedication{%
Dedication.
}

% Abstract (recommended length around 80-200 words; this is not a copy of your thesis assignment!)
\def\Abstract{%
Abstract.
}

% 3 to 5 keywords (recommended), each enclosed in curly braces
\def\Keywords{%
Top Trees, Complexity, Implementation
}

%% The hyperref package for clickable links in PDF and also for storing
%% metadata to PDF (including the table of contents).
\usepackage[pdftex,unicode]{hyperref}   % Must follow all other packages
\hypersetup{breaklinks=true}
\hypersetup{pdftitle={\ThesisTitle}}
\hypersetup{pdfauthor={\ThesisAuthor}}
\hypersetup{pdfkeywords=\Keywords}
\hypersetup{urlcolor=blue}

% Definitions of macros (see description inside)
%%% This file contains definitions of various useful macros and environments %%%
%%% Please add more macros here instead of cluttering other files with them. %%%

%%% Minor tweaks of style

% These macros employ a little dirty trick to convince LaTeX to typeset
% chapter headings sanely, without lots of empty space above them.
% Feel free to ignore.
\makeatletter
\def\@makechapterhead#1{
  {\parindent \z@ \raggedright \normalfont
   \Huge\bfseries \thechapter. #1
   \par\nobreak
   \vskip 20\p@
}}
\def\@makeschapterhead#1{
  {\parindent \z@ \raggedright \normalfont
   \Huge\bfseries #1
   \par\nobreak
   \vskip 20\p@
}}
\makeatother

% This macro defines a chapter, which is not numbered, but is included
% in the table of contents.
\def\chapwithtoc#1{
\chapter*{#1}
\addcontentsline{toc}{chapter}{#1}
}

% Draw black "slugs" whenever a line overflows, so that we can spot it easily.
\overfullrule=1mm

%%% Macros for definitions, theorems, claims, examples, ... (requires amsthm package)

\theoremstyle{plain}
\newtheorem{thm}{Theorem}
\newtheorem{lemma}[thm]{Lemma}
\newtheorem{claim}[thm]{Claim}

\theoremstyle{plain}
\newtheorem{defn}{Definition}

\theoremstyle{remark}
\newtheorem*{cor}{Corollary}
\newtheorem*{rem}{Remark}
\newtheorem*{example}{Example}

%%% An environment for proofs

%%% FIXME %%% \newenvironment{proof}{
%%% FIXME %%%   \par\medskip\noindent
%%% FIXME %%%   \textit{Proof}.
%%% FIXME %%% }{
%%% FIXME %%% \newline
%%% FIXME %%% \rightline{$\square$}  % or \SquareCastShadowBottomRight from bbding package
%%% FIXME %%% }

%%% An environment for typesetting of program code and input/output
%%% of programs. (Requires the fancyvrb package -- fancy verbatim.)

\DefineVerbatimEnvironment{code}{Verbatim}{fontsize=\small, frame=single}

%%% The field of all real and natural numbers
\newcommand{\R}{\mathbb{R}}
\newcommand{\N}{\mathbb{N}}

%%% Useful operators for statistics and probability
\DeclareMathOperator{\pr}{\textsf{P}}
\DeclareMathOperator{\E}{\textsf{E}\,}
\DeclareMathOperator{\var}{\textrm{var}}
\DeclareMathOperator{\sd}{\textrm{sd}}

%%% Transposition of a vector/matrix
\newcommand{\T}[1]{#1^\top}

%%% Various math goodies
\newcommand{\goto}{\rightarrow}
\newcommand{\gotop}{\stackrel{P}{\longrightarrow}}
\newcommand{\maon}[1]{o(n^{#1})}
\newcommand{\abs}[1]{\left|{#1}\right|}
\newcommand{\dint}{\int_0^\tau\!\!\int_0^\tau}
\newcommand{\isqr}[1]{\frac{1}{\sqrt{#1}}}

%%% Various table goodies
\newcommand{\pulrad}[1]{\raisebox{1.5ex}[0pt]{#1}}
\newcommand{\mc}[1]{\multicolumn{1}{c}{#1}}

%%% Own definitions
\renewcommand{\O}{{\cal O}}
\def\I{\it\aftergroup\/}
\def\Cpp{C{\tt ++}}


% Comment out second line to disable.
\newcommand{\TODO}[1]{}
\renewcommand{\TODO}[1]{{\color{red}\bf TODO: {#1}}}


\parskip=2pt

\input acronyms.tex

%%%%%%%%%%%%%%%%%%%%%%%%%%%%%%%%%%%%%%%%%%%%%%%%%%%%%%%%%%%%%%%%%%%%%%%%%%%%%%%%
% Title page and various mandatory informational pages
\begin{document}
%%% Title page of the thesis and other mandatory pages

%%% Title page of the thesis

\pagestyle{empty}
\hypersetup{pageanchor=false}
\begin{center}

\centerline{\mbox{\includegraphics[width=166mm,type=pdf,ext=.epdf,read=.epdf]{logo-en}}}

\vspace{-8mm}
\vfill

{\bf\Large MASTER THESIS}

\vfill

{\LARGE\ThesisAuthor}

\vspace{15mm}

{\LARGE\bfseries\ThesisTitle}

\vfill

\Department

\vfill

\begin{tabular}{rl}

Supervisor of the master thesis: & \Supervisor \\
\noalign{\vspace{2mm}}
Study programme: & \StudyProgramme \\
\noalign{\vspace{2mm}}
Study branch: & \StudyBranch \\
\end{tabular}

\vfill

% Zde doplňte rok
Prague \YearSubmitted

\end{center}

\newpage

%%% Here should be a bound sheet included -- a signed copy of the "master
%%% thesis assignment". This assignment is NOT a part of the electronic
%%% version of the thesis. DO NOT SCAN.

%%% A page with a solemn declaration to the master thesis

\openright
\hypersetup{pageanchor=true}
\pagestyle{plain}
\pagenumbering{roman}
\vglue 0pt plus 1fill

\noindent
I declare that I carried out this master thesis independently, and only with the cited
sources, literature and other professional sources.

\medskip\noindent
I understand that my work relates to the rights and obligations under the Act No.~121/2000 Sb.,
the Copyright Act, as amended, in particular the fact that the Charles
University has the right to conclude a license agreement on the use of this
work as a school work pursuant to Section 60 subsection 1 of the Copyright Act.

\vspace{10mm}

\hbox{\hbox to 0.6\hsize{%
In \hbox to 2cm{\dotfill} date \hbox to 2cm{\dotfill}
\hss}\hbox to 0.4\hsize{%
signature of the author
\hss}}

\vspace{20mm}
\newpage

%%% Mandatory information page of the thesis

\openright

\vbox to 0.5\vsize{
\setlength\parindent{0mm}
\setlength\parskip{5mm}

{\bf Title:}
\ThesisTitle

{\bf Author:}
\ThesisAuthor

{\bf \DeptType:}
\Department

{\bf Supervisor:}
\Supervisor, \SupervisorsDepartment

{\bf Abstract:}
\Abstract

{\bf Keywords:}
\Keywords

\vss}

\newpage

%%% Dedication

\openright

\noindent
\Dedication

\newpage

\openright
\pagestyle{plain}
\pagenumbering{arabic}
\setcounter{page}{1}


%%% A page with automatically generated table of contents of the master thesis

\tableofcontents

\listoftodos

%%% Each chapter is kept in a separate file
\chapter*{Introduction}
\addcontentsline{toc}{chapter}{Introduction}

Main aim of this thesis is to provide two different {\I Top Trees}
implementations and to compare them in different situations. Both implementation
were written from scratch in \Cpp{} to provide comparable results.

{\I Top Trees} are not so well known data structure which could be used to
maintain information of some dynamically updated collection of trees. User of
this data structure defines four basic operations, which are used internally
when Top Trees structure is changing. When there occurs some cutting or joining
on underlying trees the structure updates internally stored information using
these user functions.

This data structure could be used for example to dynamically maintain diameter,
center or median (minimizing weighted distance from all other vertices) of given
tree in time $\O(\log N)$ (where $N$ denotes the number of vertices).

Because it is essential to understand how the Top Trees structure works, some
basic principles of the Top Trees structure are introduced in the
\Cref{chap:TopTrees} and some basic principles of Topology trees used in one of
the implementations are introduced in the \Cref{chap:TopologyTrees}. Some
examples of problems, which could Top Trees handle quickly, are listed in
\Cref{chap:Problems} of this thesis.

Basic usage of both implementations and some technical details are the contents
of the \Cref{chap:Implementation}. \Cref{chap:ImplementationSelfAdjusting} and
\Cref{chap:ImplementationTopology} describes details of both implementations.

First implementation of the Top Trees structure is based on article {\I
Self-Adjusting Top Trees} \cite{SelfAdjustingTT} by Tarjan and Werneck. This
implementation promises quick amortized time per operation (with small
constant), but it does not guarantee these times in worst case. This
implementation is described in \Cref{chap:ImplementationSelfAdjusting} of this thesis.

Second implementation is based on article {\I Maintaining Information in Fully-
Dynamic Trees with Top Trees} \cite{TopTrees} by Alstrup, Holm, Lichtenberg and
Thorup and uses Topology trees introduced by Frederickson in
\cite{DSforDynamicallyMaintainingRootedTrees}. This implementation promises time
$\O(\log N)$ in worst-case but with much larger multiplicative constant. This
implementation is described in \Cref{chap:ImplementationTopology} of this thesis.

To compare both implementations it was necessary to perform some experiments on
different problems on different graphs with different sizes. Experiments were
performed on problem of {\I maximum edge weight in tree with interval updates}
(described in section \ref{sec:maximum_edge_weight}) and on problem of {\I edge
2-connectivity} (described in section \ref{sec:edge_2_connectivity}. Details of
these experiments and their setup are described in \Cref{chap:Experiments}.

We expected that the first implementations would have smaller multiplicative
constant than the second one. This expectation turned out to be right and
multiplicative constant for both implementations was measured in
\Cref{chap:Results} together with some results for turning out unnecessary
updates during some operation in the second implementation.

\chapter{Top Trees}

Top Trees are data structure intended to maintain informations of underlying
dynamically updated forest. They were introduced by Alstrup et al. in 2003 \cite{TopTrees}.

\section{Definition}

{\I Top Trees structure} acts as driver for underlying forest. It represents
underlying trees as collection of generalized edges called {\I clusters}. Each
{\I Cluster} represents some continuous subtree in the underlying forest. Only
some of them called {\I root clusters} (which represents whole trees of the
underlying forest) could be directly accessed by the user.

User defines format of the data stored in these clusters and four basic
functions {\sc Create}, {\sc Destroy}, {\sc Join} and {\sc Split} used to
manipulate with clusters data. Above that user could define fifth function
{\sc Choose} which is needed for some use cases but it is not needed for basic
usage.

Then user controls the Top Trees structure by using operations $\cut{u,v}$,
$\link{u,v}$ and $\expose{u,v}$. Last of them
makes cluster representing the path between vertices $u$ and $v$ a root cluster
(because root clusters are the only clusters of the top tree, whose could be
accessed by the user). The Top Trees structure dynamically updates stored data
in clusters by using user defined functions.


%%%%%%%%%%%%%%%%%%%%%%%%%%%%%%%%%%%%%%%%%%%%%%%%%%%%%%%%%%%%%%%%%%%%%%%%%%%%%%%%


\section{Clusters}

As has been said {\I Clusters} are generalized edges. Each cluster has two
{\I boundary vertices} and represents part of the underlying forest between
these vertices.

Clusters in the Top Trees structure are connected into binary trees where each
leaf represents one edge of the underlying forest and each inner vertex represents
contraction of its children. More about this structure will be discussed
later in {\I Cluster model} subsection. Before that we need to introduce types
of clusters.

There are three types of clusters:

\begin{itemize}

\item {\bf Base cluster} -- represents one edge of the underlying forest (and
each edge of the underlying forest has exactly one base cluster, it is 1:1 mapping),
boundary vertices are endpoints of the edge.

This cluster could appear only as leaf in the Top Trees structure.

\item {\bf Rake cluster} -- represents one way how to contract two clusters
with one common boundary vertex. Lets have two clusters $C_1(u,v)$ and
$C_2(v,w)$ next to each other around common boundary vertex $v$ (and lets the
$C_1$ be the left one of them in some topological order given for example by
indexes of the edges).

Then we construct {\I left rake cluster} by {\I raking} the left cluster ($C_1$)
on the right one ($C_2$). The resulting cluster would have the same boundary
vertices as the cluster $C_2$.

Or we can construct {\I right rake cluster} in the same way -- by {\I raking}
the right cluster ($C_2$) on the left one ($C_1$). The resulting cluster would
have the same boundary vertices as the cluster $C_1$.

\todo{Image of rake cluster}

\item{\bf Compress cluster} -- represents other contraction of the two clusters
with one common boundary vertex $v$ into one cluster by attaching first cluster
after the other. Right before compressing the common vertex $v$ must have degree
(number of incident clusters) exactly two. If there are other clusters attached
to the same common boundary vertex they must be firstly {\I raked} onto one of
the compressed clusters.

If boundary vertices of the cluster $C_1$ were $(u,v)$ and boundary vertices
of the cluster $C_2$ were $(v,w)$, the cluster $C=compress(C_1,C_2)$ would have
boundary vertices $(u,w)$ (and we will call it {\I compress cluster
of vertex $v$} and the operation {\I compressing around vertex $v$}).
This cluster also in some way represents the vertex $v$ and we will use it as
{\I handle} of this vertex.

\todo{Image of compress cluster}

\end{itemize}

\subsection{Clusters model}

Clusters in the Top Trees structure are organized into binary trees. Leaves of
these trees (Base clusters) represents edges of the underlying trees and each
inner vertex represents contraction of two child clusters into one.

Compress and rake clusters have each of them two children, base clusters are
childless. Each cluster represent continuous subtree of the underlying forest.
By combination of clusters we could represent each underlying tree as one {\I
root cluster} (we will sometime refer to the representation of one continuous
underlying tree as one {\I top tree}).

Compress clusters are used to represent paths in the underlying tree -- each
path could be compressed into one {\I compress tree} consisting only of compress
clusters. If there are branches separating from this path, they are firstly
recursively represented as single clusters ({\I rake trees}) and then they are
{\I raked onto} clusters in the path.

Because there are $N$ base clusters for an underlying tree with $N$ edges and
each inner vertex of the corresponding top tree joins two adjacent clusters into
one, there will be $N-1$ inner clusters for representing this underlying tree.

Underlying tree could have (and usually have) many different divisions into
paths and so the underlying tree have many different representations. Crucial
part of the top trees structure is to maintain this representation in some nice
form during updates.

\subsection{Extended clusters model}

Tarjan and Werneck in \cite{SelfAdjustingTT} suggested that in some cases it may
be useful to modify structure of the clusters and they introduced
{\I foster children} for {\I compress clusters}. In their suggestion a compress
cluster could have up to four descendants -- two normal children and up to two
foster children.

Normal children of a compress cluster are clusters from the compressed path and
foster children are clusters originating from the separating branches. In normal
cluster model they would be raked onto clusters from path and the path would be
compression of these rake clusters.

In this extended model the clusters originating from the separating branches are
firstly combined in so called {\I rake trees} -- there are maximally two rake
trees around each path vertex, one of them is raked from branches on one side of
the path and the second one is raked from branches on the other side of the
path. And these rake trees are connected as left and right foster child of the
compress cluster constructed from this part of the path.

\todo{Image of rake trees around path}

During computation ({\sc Join} and {\sc Split} operations) there is need to use
virtual rake clusters, but it is only $\O(1)$ time complexity per one compress
cluster. We will discuss it later in the first implementation for which this
extended model is used.

%%%%%%%%%%%%%%%%%%%%%%%%%%%%%%%%%%%%%%%%%%%%%%%%%%%%%%%%%%%%%%%%%%%%%%%%%%%%%%%%


\section{User defined functions}

There are four basic function to manipulate with the clusters data which have to
be implemented by users of the Top Trees structure. Then user uses public Top
Trees structure operation and these user functions are used internally when
constructing, destroying or reorganizing clusters.

If user wants to use the {\I Search} operation, he has to implement fifth user
function {\I Choose} to traverse around the path by choosing children.

Examples of the functions and related problems are given in the Chapter 2.

\subsection{Create}

This function is called when new base cluster is created. It gets reference to
the underlying edge and to the newly created base cluster, populates base
clusters data based on the underlying edge and runs others user defined
operations according to logic of given problem.

\subsection{Destroy}

Opposite of the Create function. This function is called just before deleting
base cluster. It gets reference to the underlying edge and to the base cluster
which would be destroyed and it could perform some end-of-life operations (like
saving computed data from the cluster).

\subsection{Join}

This function is called during connecting two clusters into one (compress or rake)
cluster. It gets references to both of the connecting clusters and to the parent
cluster and it should aggregate data from children to the parent or perform other
join-related operations according to logic of given problem.

\subsection{Split}

Opposite of the Join function. It is called just before removing connection
between parent cluster and its children. This function gets reference to the
parent cluster and both of its children and it should distribute data from the
parent into children.

The Join and Split operations are frequently called during reorganization of the
Top trees structure -- common pattern is to Split everything around changed path
in the top-down manner, reorganize the structure and then Join everything in the
bottom-up manner.

\subsection{Choose}

This operation for given cluster selects one of its child clusters. It gets
reference to the cluster and its children and returns reference to one of them.
It is used internally by the {\I Search} operation.

%%%%%%%%%%%%%%%%%%%%%%%%%%%%%%%%%%%%%%%%%%%%%%%%%%%%%%%%%%%%%%%%%%%%%%%%%%%%%%%%


\section{Top Trees operations}

These are the only operations which could user use to manipulate with the Top
Trees structure. In addition to that user could access root clusters and read
informations from them.

\subsubsection{Handles}

Following operations are defined for pair of vertices of the underlying forest,
but the Top Trees structure operates on (generalized) edges. We need to map
these vertices to clusters.

We want to choose clusters whose in some way represents operations with
vertices. Every cluster represents some path and for given vertex we want to
choose cluster which has this vertex in its path. Also we want that the chosen
cluster could be easily transformed around this vertex. This means that we want
cluster that has chosen vertex as its boundary vertex or common vertex
(if compress clusters).

To accomplish this mapping we define {\I handle} for each vertex of the
underlying forest in this way:

\begin{itemize}
\item Isolated vertex has no handle.
\item If the vertex is leaf of the underlying tree the
handle for this vertex is the topmost compress (or base) cluster having this
vertex as one of its endpoints.
\item And finally if the vertex has degree at least two the compress cluster of
this vertex (compress cluster having this vertex as the common boundary vertex)
is the handle of this vertex.
\end{itemize}

One node could be handle for at most three vertices -- two as endpoints and one
as common boundary vertex. To mark handle of a vertex $v$ we will use notation
$N_v$.

With handles we could transform operations with vertices into operations with
clusters.

\subsection{Expose}

This is the most basic operation (which is used internally by others
operations). Calling $\expose{u,v}$ will result into several changes in the Top
Trees structure depending on the positions of vertices $u$ and $v$.

Implementation of the expose slightly differ in the first and the second
implementation (first implementation uses splays and splices), but the basic
principle is the same.

\subsubsection{Soft expose}

If given vertices are in different components (they are not connected by path) both the
handles of $u$ and $v$ are brought to the roots of corresponding top trees
(similarly if $u=v$).

When they are in the same component (they are connected by path) firstly the
handle of $u$ is brought to the root of corresponding top tree. If the same
cluster is also the handle of $v$ we are done, otherwise the handle of $v$ is
brought as close to the root as possible (but not replacing the handle of $u$ as
root).

\subsubsection{Hard expose}

Ideal situation is when $u$ and $v$ are endpoints of the root cluster, it is
possible when both of them have degree one. But in general root cluster could
represent some path $(x,y)$ with path $(u,v)$ as subpath. In this case we need
to temporarily convert ends of this path (paths $(x,u)$ and $(v,y)$) into rake
clusters so the compress tree would represent the path $(u,v)$ with these ends
raked onto this path.

\todo{More detailed info about hard expose.}

\subsection{Cut}

Operation $\cut{u,v}$ deletes edge between vertices $u$ and $v$ and reorganizes
the Top Trees structure to reflect this change. We assume that $u\ne v$ and
there exists edge $(u,v)$.

Firstly the structure internally performs {\I soft expose} on $u,v$ to bring
theirs handles to the top of corresponding top tree. In this situation the
handle of $u$ will be the root cluster and the handle of $v$ will be child of
the root cluster.

We have to destroy the base node representing $(u,v)$ and remove connection
between these two handles, some reorganization of child clusters is needed (it
is implementation specific). This operation results in two top trees, one with
handle of~$u$ as root cluster and second with handle of~$v$ as root cluster.

\todo{More detailed info about cut.}

\subsection{Link}

Calling $\link{u,v}$ on two disjoint vertices joins them by new edge $(u,v)$.
This is similar to the cut operation but in the opposite way. Firstly we start
by {\I exposing} (bringing to root) handles of $u$ and $v$.

Then we create new base cluster representing edge $(u,v)$, connect it on the
right place into one of the top trees, connect this tree as child of the root
cluster of the second tree and reorganize clusters (it is implementation
specific).

\todo{More detailed info about link.}

\subsection{Search}

When defined the {\I Choose} user function this operation could be used to find
and return specific cluster.

\todo{Example of search.}

\chapter{Amortized implementation of Top Trees}

\chapter{Amortized implementation of Top Trees}

This implementation is based on article Self-Adjusting Top Trees
\cite{SelfAdjustingTT} by Tarjan and Werneck and uses the extended clusters
model with foster children discussed in the previous chapter.

\section{Construction}

Tarjan and Werneck in \cite{SelfAdjustingTT} suggested this construction:
\begin{enumerate}
\item Choose root $r$ as a vertex with degree one.
\item Orient all edges in the tree containing vertex $r$ towards the $r$.
\item Divide tree into paths starting in some leaf and continuing along the
direction of the edges -- the first path will end in the root $r$ and became the
{\I root path}, other paths ends in some other path.
\item Recursively compute clusters to represent each path incident to the
root path and create {\I rake trees} from these incident paths.
\item Create binary tree of compress clusters to represent the root path and
connect rake trees as foster children.
\item If there are some unused vertices of degree one start the process again
from any of these vertices to construct another top tree.
\end{enumerate}

In my implementation I choose equivalent construction but in more recursive
manner. I started the same way by choosing the root $r$ as vertex with degree
one, but I don't divide the tree into paths.

Starting from the second vertex we choose one neighbour as continuation of the
path and recursively called the same function on all other neighbours. Recursion
returns cluster representing for each of the subtrees and then they could be
raked into left and right rake trees and saved into this vertex for future use.

When compressing the path into compress clusters we just look into the common
vertex of compressed clusters and if there are saved rake trees we connect them
as left and right foster children.

This construction is easier to implement and gives us ability to better control
the shape of the resulting top tree. By choosing neighbors instead of directing
paths from leafs we could prefer longer paths by choosing deepest neighbors
(we firstly run \gls{dfs} to obtain depths). Longer paths are better contracted
in binary tree structure of compress clusters to obtain lower top tree.

\section{Expose -- splaying and splicing}

Expose in this implementation is based on splaying and splicing whose are used
to bring handles of given vertices to the top of theirs top trees.

Because of extended clusters model each top tree consist from independent
{\I compress trees} (only compress clusters as internal nodes) and
{\I rake trees} (only rake clusters as internal nodes).

Whole top tree is one compress tree (which represents the {\I root path}). It
has base clusters as leafs and root of rake trees as foster children (these
foster children are other paths connected to the root path). These rake trees
have rake clusters as theirs internal nodes and base or compress clusters as
theirs leaf. And so on.

This division of the top tree into smaller blocks could be used to expose given
pair of vertices in the root of the top tree. We will use operations of
{\I splay} and {\I splice} introduced by Sleator and Tarjan in
\cite{SelfAdjustingBST}.

\subsection{Split and Join operations}

Before doing any operation that changes shape of a top tree all nodes involved
in this operation must be splitted (and all theirs parent on the way to the root
of this tree too). This is crucial because after changing shape of the top tree
a data stored in these nodes may be changed (for example depth of subtree bellow
this node).

Split operations has to be done in top-down manner (starting from the root). The
easiest way how to accomplish this is to have flag in each node if it is
splitted and recursively split parent before splitting current node. All
splitted nodes should be logged into some list to easily join all of them after
completing current operation.

Joining is done in opposite direction, in bottom-up manner (ending in the root).
We will assume that before doing anything with any node during splaying and
splicing operations we firstly split this node and after completing the entire
expose operation we will call join on all splitted nodes.

\subsection{Splaying}

Splaying is originally a heuristic for balancing binary trees which uses an idea
that often used nodes should be near the root of the tree. Each operation (find,
delete, \dots) on a vertex in splay tree is preceded by splaying this vertex
which moves this vertex to the root of the tree.

Splaying is done by rotations or double rotations (which are called {\I zig-zig}
or {\I zig-zag} rotations). That moves target vertex up by one or two levels
leaving all vertices in the right order.

Although some not so often used vertices may be in $\O(n)$ distance from the
root, Sleator and Tarjan in \cite{SelfAdjustingBST} proved that all operations
works in amortized time $\O(\log n)$ per operation.

In the Top Trees structure We will use {\I guarded splays} that works exactly
the same way as normal splays, but it stops splaying when reaching a guard (some
node). Normal splay has as guard root of the whole tree, but we want to do
splays limited only inside compress or rake tree (to not mixing compress and
rake clusters).

Implementation of the splay is straightforward. Only noticeable detail (which
Tarjan and Werneck mentioned in the \cite{SelfAdjustingTT}) is that foster
children are not affected by any rotations, they always keeps the same parents.

\subsection{Splicing}

Only by splaying we are not able to expose a path containing both given
vertices. And we are not even able to carry compress clusters over rake trees
(because rake trees are connected as foster children under compress trees). For
that we need to change partitioning of the top tree into paths.

Lets take some vertex $v$ which is internal to some path $a,b$ (so there is
compress cluster around this vertex) and which has node representing path $c,v$
in its connected rake tree. We may change partitioning into paths by removing
one half of the original path (for example $a,v$), connect it into rake tree,
removing node representing path $c,v$ from the rake tree and changing the
compress cluster that it now represents path $c,b$.

\todo{Picture of splice}

Splicing is used after splaying and its task is to move cluster, which is leaf
of some rake tree, to the compress tree above it.

Tarjan and Werneck in the \cite{SelfAdjustingTT} only described the main idea in
the general case, but implementations details had to be worked out. I decided
to do {\I left splice} (replacing left child of the compress cluster) which was
original choose of Tarjan and Werneck, but the procedure would work the same
way if we choose to replace the right child.

My implementation does splicing in this way:
\begin{enumerate}
\item Prepare empty lists of left and right neighbors.
\item Starting from given node $N$ go up until reaching node $N_c$ in compress
tree. During that add left and right neighbors into neighbors lists and destroy
old internal nodes of these rake trees.
\item Add left child of the $N_c$ to the appropriate list of neighbors and
connect $N$ as the new left child of the $N_c$ (that makes $N$ part of the above
compress tree).
\item Construct new left and right foster children (rake trees) from left and
right neighbors lists with respecting their order. New internal rake nodes have
to be constructed (and added into list of nodes for joining).
\end{enumerate}

To assure that we not broke connectivity of $N_c$ with above clusters by
replacing left child we need to check children of the $N_c$. If the above
cluster (parent) is compress cluster, one of boundary vertices is common vertex
of the parent. And if the parent is rake cluster, one of boundary vertices is
the vertex around which is the parent raked. If this connection is in the left
child we need to flip children (otherwise child with the connection would be
moved into rake tree and connection would broke).

This detail was not discussed by Tarjan and Werneck -- they only suggested
transforming top trees into some normalized forms whose would work for both
splaying and splicing. But they not mentioned some corner cases where such
checking and flipping is needed.

\subsection{Soft expose}

Soft expose is the first part of the expose operation. It is called as
{\I soft\_expose(u,v)} and its target is to bring path $u,v$ as subpath into
the root cluster of corresponding top tree (so after soft expose there should
be some path $a\dots u\dots v\dots b$ in the root cluster). Truncating the root
cluster to contain only $u,v$ path is quest for hard expose.

Soft expose takes handles of both vertices and brings them to the top of theirs
top tree. Procedure for one vertex (as described by Tarjan and Werneck in
\cite{SelfAdjustingTT}) is following:

\begin{enumerate}
\item Local Splays -- Starting from the handle of given vertex:
  \begin{enumerate}
  \item Splay current node inside compress tree (that makes it root of that compress tree)
  \item If reaching root clusters (there is no parent) $\rightarrow$ stop the cycle.
  \item Splay parent inside rake tree (parent of a root of a compress tree is always rake cluster).
  \item Take parent of that node (it is compress cluster having this rake tree as foster child)
  and repeat the cycle.
  \end{enumerate}
\item Splices -- Starting from the handle of given vertex splice current vertex and
moving to its parent. That in every step moves handle of the given vertex across
one rake tree to the above compress tree. Finally it moves this handle into the
root compress tree.
\item Global Splay -- Perform splay on the handle of given vertex (now in the
root compress tree) to make it root of this top tree.
\end{enumerate}

When exposing the second handle (handle of the second vertex) only difference
is that all splays are guarded by the first handle (to assure that it remains
at the top).

Tarjan and Werneck discussed when one of the vertices has degree one (and its
handle is not a compress cluster around this vertex), in this case they realized
that upper procedure leads to exposing handle of the second vertex with first
vertex as one of its endpoints. They not discussed situation when both vertices
have degree one -- in that case before starting procedure for the second vertex
we have to ensure that base cluster of the first vertex is not the left child
of root cluster (otherwise it would be moved to rake tree when splicing the
handle of the second vertex).

And what are results of this procedure? When both vertices are in different top
trees this procedure brings both handles to the top of corresponding top trees.
When they are in the same top tree this procedure leads to exposing cluster
which contains

\subsection{Hard expose}

Hard expose is the second part of the expose procedure, it follows the soft
expose and its job is to truncate the path in the root cluster that it contains
only the exposed subpath.

\todo{Hard expose procedure}

\chapter{Worst-case implementation of Top Trees}

\chapter{Experiments}


\chapter*{Conclusion}
\addcontentsline{toc}{chapter}{Conclusion}


%%% Bibliography
%%% Bibliography (literature used as a source)
%%%
%%% We employ bibTeX to construct the bibliography. It processes
%%% citations in the text (e.g., the \cite{...} macro) and looks up
%%% relevant entries in the bibliography.bib file.
%%%
%%% The \bibliographystyle command selects, which style will be used
%%% for references from the text. The argument in curly brackets is
%%% the name of the corresponding style file (*.bst). Both styles
%%% mentioned in this template are included in LaTeX distributions.

% \bibliographystyle{plainnat}    %% Author (year)
\bibliographystyle{unsrt}     %% [number]

\renewcommand{\bibname}{Bibliography}

%%% Generate the bibliography. Beware that if you cited no works,
%%% the empty list will be omitted completely.

\bibliography{bibliography}

%%% If case you prefer to write the bibliography manually (without bibTeX),
%%% you can use the following. Please follow the ISO 690 standard and
%%% citation conventions of your field of research.

% \begin{thebibliography}{99}
%
% \bibitem{lamport94}
%   {\sc Lamport,} Leslie.
%   \emph{\LaTeX: A Document Preparation System}.
%   2nd edition.
%   Massachusetts: Addison Wesley, 1994.
%   ISBN 0-201-52983-1.
%
% \end{thebibliography}


%%% Figures used in the thesis (consider if this is needed)
% \listoffigures

%%% Tables used in the thesis (consider if this is needed)
%%% In mathematical theses, it could be better to move the list of tables to the beginning of the thesis.
% \listoftables

%%% Abbreviations used in the thesis, if any, including their explanation
%%% In mathematical theses, it could be better to move the list of abbreviations to the beginning of the thesis.
\addcontentsline{toc}{chapter}{List of Abbreviations}
\advance\glsdescwidth by 2cm
\printglossary[type=\acronymtype,title={List of Abbreviations}]


%%% Attachments to the master thesis, if any. Each attachment must be
%%% referred to at least once from the text of the thesis. Attachments
%%% are numbered.
%%%
%%% The printed version should preferably contain attachments, which can be
%%% read (additional tables and charts, supplementary text, examples of
%%% program output, etc.). The electronic version is more suited for attachments
%%% which will likely be used in an electronic form rather than read (program
%%% source code, data files, interactive charts, etc.). Electronic attachments
%%% should be uploaded to SIS and optionally also included in the thesis on a~CD/DVD.
\chapwithtoc{Attachments}

{\bf Attachment 1:} Source code of the two implementations (including experimental
data and generators)

\openright
\end{document}
