\chapter{Top Trees}
\label{chap:TopTrees}

Top Trees are data structure intended to maintain informations of underlying
dynamically updated forest. They were introduced by Alstrup, Holm, Lichtenberg
and Thorup in {\I Minimizing Diamaters of Dynamic Trees}
\cite{MinimizingDiamatersOfDynamicTrees} in 1997 as variant of the Topology
Trees and they were extended by the same authors in
{\I Maintaining Information in Fully-Dynamic Trees with Top Trees} \cite{TopTrees} in 2003.

\section{Definition}

{\I Top Trees structure} acts as driver for underlying forest. It represents
underlying trees as collection of generalized edges called {\I clusters}. Each
{\I Cluster} represents some subtree in the underlying forest. Only
some of them called {\I root clusters} (which represents whole trees of the
underlying forest) could be directly accessed by the user.

User defines format of the data stored in these clusters and four basic
{\I user functions} \Create, \Destroy, \Join{} and \Split{} used to
manipulate with clusters data. Above that user could define fifth function
\Choose{} which is needed for some use cases but it is not needed for basic
usage.

Then user controls the Top Trees structure by using {\I operations} $\Cut(u,v)$,
$\Link(u,v)$ and $\Expose(u,v)$. Last of them
makes cluster representing the path between vertices $u$ and $v$ a root cluster
(because root clusters are the only clusters of the top tree, whose could be
accessed by the user). The Top Trees structure dynamically updates stored data
in clusters by using user defined functions.

{\I Notation: We will use capitalize form to denote situations where we refer
directly to the defined user functions or to the top trees operations called by
users. When referring to the generic process of joining, splitting or to the
internal procedures related to these processes we will use normal font style.}

%%%%%%%%%%%%%%%%%%%%%%%%%%%%%%%%%%%%%%%%%%%%%%%%%%%%%%%%%%%%%%%%%%%%%%%%%%%%%%%%


\section{Clusters}

As has been said {\I Clusters} are generalized edges. Each cluster has two
{\I boundary vertices} and represents part of the underlying forest between
these vertices. We denote two clusters as {\I connected} if they are edge
disjoint and they share one boundary vertex.

A {\I Clusterization} is division of the underlying forest into clusters such
that each edge is in exactly one cluster. As we mentioned above {\I roots
clusters} are the ones that represent whole trees in the underlying forest
(all edges of their underlying tree are contracted inside and there are no
outgoing edges -- this means that in a clusterization both their boundary
vertices are not connected with any other cluster).

Another special clusters are {\I leaf clusters}. We denote a cluster as a
{\I leaf cluster} in some clusterization if only one of its boundary vertices is
connected to another cluster.

Clusters in the Top Trees structure are organized into binary trees (called
{\I top trees}) where each leaf represents one edge of the underlying forest and
each inner vertex represents contraction of its children. More about this
structure will be discussed later in the {\I Cluster model} subsection. Before
that we need to introduce types of clusters. There are three types of clusters:
\begin{itemize}

\item {\bf Base cluster} -- represents one edge of the underlying forest (and
each edge of the underlying forest has exactly one base cluster, it is 1:1 mapping),
boundary vertices are endpoints of the edge.
This cluster could appear only as leaf in the Top Trees structure.

\item {\bf Rake cluster} -- represents one way how to contract two clusters
with one common boundary vertex. Let's have two clusters $C_1(u,v)$ and
$C_2(v,w)$ next to each other around common boundary vertex $v$ (and let the
$C_1$ be the left one of them in some topological order given for example by
indices of the edges or by some planar embedding).

If the left cluster ($C_1$) is a leaf cluster then we can construct {\I left
rake cluster} by {\I raking} the left cluster ($C_1$) on the right one ($C_2$).
The resulting cluster would have the same boundary vertices as the cluster $C_2$.

If the right cluster ($C_2$) is a leaf cluster then we can, similarly to the
previous case, construct {\I right rake cluster} by {\I raking} the right
cluster ($C_2$) on the left one ($C_1$). The resulting cluster would have the
same boundary vertices as the cluster $C_1$.

\begin{figure}[h]
\centering
\asyinclude{pic/chap01_rake.asy}
\caption{Left and right rake clusters}
\end{figure}

\item{\bf Compress cluster} -- represents other contraction of the two clusters
with one common boundary vertex $v$ into one cluster by attaching first cluster
after the other. Right before compressing the common vertex $v$ must have degree
(number of incident clusters) exactly two. If there are other clusters attached
to the same common boundary vertex they must be firstly {\I raked} onto one of
the compressed clusters.

If boundary vertices of the cluster $C_1$ were $(u,v)$ and boundary vertices
of the cluster $C_2$ were $(v,w)$, the cluster $C=compress(C_1,C_2)$ would have
boundary vertices $(u,w)$ (and we will call it {\I compress cluster
of vertex $v$} and the operation {\I compressing around vertex $v$}).
This cluster also in some way represents the vertex $v$ and we will use it as
{\I handle} of this vertex.

\begin{figure}[H]
\centering
\asyinclude{pic/chap01_compress.asy}
\caption{Compress cluster}
\end{figure}

\end{itemize}

\subsection{Clusters model}

Clusters in the Top Trees structure are organized into binary trees. Leaves of
these trees (Base clusters) represent edges of the underlying trees and each
inner vertex represents contraction of two child clusters into one.

Compress and rake clusters have each of them two children, base clusters are
childless. Each cluster represent subtree of the underlying forest. By
combination of clusters we could represent each underlying tree as one {\I root
cluster}. This whole binary tree of cluster contractions leading to the one root
cluster is called {\I top tree}.

Compress clusters are used to represent paths in the underlying tree -- each
path could be compressed into one {\I compress tree} consisting only of compress
clusters. If there are branches separating from this path, they are firstly
recursively represented as single clusters ({\I rake trees}) and then they are
{\I raked onto} clusters in the path.

Because there are $M$ base clusters for an underlying tree with $M$ edges and
each inner vertex of the corresponding top tree joins two adjacent clusters into
one, there will be $M-1$ inner clusters for representing this underlying tree.

Underlying tree could have (and usually have) many different divisions into
paths and so the underlying tree have many different representations. Crucial
part of the top trees structure is to maintain this representation in some nice
form during updates.

\begin{figure}[H]
\centering
\asyinclude{pic/chap01_top_tree.asy}
\caption{Original tree and corresponding top tree (rake clusters are grey)}
\end{figure}

\subsection{Extended clusters model}

Tarjan and Werneck in \cite{SelfAdjustingTT} suggested that in some cases it may
be useful to modify structure of the clusters and they introduced
{\I foster children} for {\I compress clusters}. In their suggestion a compress
cluster could have up to four descendants -- two normal children and up to two
foster children.

Normal children of a compress cluster are clusters from the compressed path and
foster children are clusters originating from the separating branches. In normal
cluster model they would be raked onto clusters from path and the path would be
compression of these rake clusters.

In this extended model the clusters originating from the separating branches are
firstly combined in so called {\I rake trees} -- there are maximally two rake
trees around each path vertex, one of them is raked from branches on one side of
the path and the second one is raked from branches on the other side of the
path. And these rake trees are connected as left and right foster child of the
compress cluster constructed from this part of the path.

\begin{figure}[h]
\centering
\asyinclude{pic/chap01_rake_trees.asy}
\caption{Rake trees (triangles) around a path, they can be connected as
foster children to compress clusters}
\end{figure}

During computation (\Join{} and \Split{} operations) there is need to use
virtual rake clusters, but it is only $\O(1)$ time complexity per one compress
cluster. We will discuss it later in the first implementation for which this
extended model is used.

\begin{figure}[h]
\centering
\asyinclude{pic/chap01_top_tree_extended.asy}
\caption{Original tree and corresponding top tree with extended clusters model
(foster children are connected by dotted edges)}
\end{figure}

%%%%%%%%%%%%%%%%%%%%%%%%%%%%%%%%%%%%%%%%%%%%%%%%%%%%%%%%%%%%%%%%%%%%%%%%%%%%%%%%


\section{User defined functions}

There are four basic function to manipulate the clusters data which have to
be implemented by users of the Top Trees structure. Then user uses public Top
Trees structure operations and these user functions are used internally when
constructing, destroying or reorganizing clusters.

If user wants to use the \Search{} operation, he has to implement fifth user
function \Choose{} to traverse around the path by choosing children.

Examples of the functions and related problems are given in the \Cref{chap:Problems}.

\subsection{\sc Create}

This function is called when new base cluster is created. It gets reference to
the underlying edge and to the newly created base cluster, populates base
cluster's data based on the underlying edge and runs other user defined
operations according to logic of given problem.

\subsection{\sc Destroy}

Opposite of the \Create{} function. This function is called just before deleting
base cluster. It gets reference to the underlying edge and to the base cluster
which would be destroyed and it could perform some end-of-life operations (like
saving computed data from the cluster).

\subsection{\sc Join}

This function is called during contraction of two clusters into one (compress or
rake) cluster. In the general view it should populate parent cluster with the
data aggregated from contracted child clusters or perform other join-related
operations according to logic of given problem.

It gets a references to both of the contracted clusters and to the newly created
parent cluster with information about their boundary vertices. From boundary
vertices of the parent and both children can be clearly determined if it is
compress or rake cluster (and which one of the children in rake cluster is raked
onto the another one) -- user may or may not use this information according to
the logic of given problem.

\subsection{\sc Split}

Opposite of the \Join{} function. It is called just before removing connection
among parent cluster and its children. This function gets reference to the
parent cluster and both of its children with information about their boundary
vertices. It should distribute data from the parent into children -- notice that
no data could be stored in the parent cluster after the Split operation (because
the parent cluster is deleted after this operation).

The \Join{} and \Split{} operations are frequently called during reorganization of the
Top trees structure -- common pattern is to split everything around changed path
in the top-down manner, reorganize the structure and then join everything in the
bottom-up manner.

\subsection{\sc Choose}

%This operation for given cluster with full information (all clusters above it
%are splitted -- efficiently this is a root cluster) selects one of its child
%clusters. It gets reference to the cluster and its children and returns
%reference to one of them. It is used internally by the \Search{} operation.

This operation for given root cluster selects one of its child clusters. It gets
reference to the cluster and its children and returns reference to one of them.
It is used internally by the \Search{} operation.

%%%%%%%%%%%%%%%%%%%%%%%%%%%%%%%%%%%%%%%%%%%%%%%%%%%%%%%%%%%%%%%%%%%%%%%%%%%%%%%%

\penalty-5000 % PRINTHACK

\section{Top Trees operations}

These are the only operations which could user use to manipulate the Top Trees
structure. In addition to that, user could access root clusters and read
informations from them.

\subsubsection{Normalized shape}

Depending on implementation there could be defined a normalized shape of the
Top Trees structure. All operations expect the Top Trees structure in this
normalized shape.

Some operations may corrupt this normalized shape and in that case the correct
shape must be restored prior to the next operation. For both following
implementations an example of such operation is the \Expose{} operation (see
implementation details).

We cannot restore the correct shape right after finishing the operation because
user may need to interact with the Top Trees structure in this corrupted state.
Therefore we have to record that the structure is in corrupted form and we have
to do check and eventually restoration at beginning of each operation. See the
following {\I Restore} operation for more details.

\subsubsection{Handles}

Following operations are defined for pair of vertices of the underlying forest,
but the Top Trees structure operates on (generalized) edges. We need to map
these vertices to clusters.

We want to choose clusters whose in some way represent operations with
vertices. Every cluster represents some path and for given vertex we want to
choose cluster which has this vertex in its path. Also we want that the chosen
cluster could be easily transformed around this vertex. This means that we want
cluster that has chosen vertex as its boundary vertex or common vertex
(if compress clusters).

To accomplish this mapping we define {\I handle} for each vertex of the
underlying forest in this way:

\begin{itemize}
\item Isolated vertex has no handle.
\item If the vertex is a leaf of the underlying tree the
handle for this vertex is the topmost compress (or base) cluster having this
vertex as one of its boundary vertices (rake clusters cannot be handles).
\item Root cluster is handle for its boundary vertices regardless of their
degree in the underlying tree.
\item And finally if the vertex has degree at least two the compress cluster of
this vertex (compress cluster having this vertex as the common boundary vertex)
is the handle of this vertex.
\end{itemize}

One node could be handle for at most three vertices -- two as endpoints and one
as common boundary vertex. To mark handle of a vertex $v$ we will use notation
$N_v$.

With handles we could transform operations with vertices into operations with
clusters.

\subsection{\sc Expose}

Expose is one of the most basic operation. Calling $\Expose(u,v)$ will result in
exposing the $u\dots v$ path (if exists) in the root cluster, which the user
could modify.

Implementation of the expose slightly differs in the first and the second
implementation (first implementation uses splays and splices and the second one
does expose through several splits and joins), but result of both is the
same root cluster (but with possibly different decomposition to subclusters).

See details of both implementations for more information.

\subsection{\sc Restore}

As we mentioned before we may need to keep the Top Trees structure in some
normalized shape. Purpose of this operation is to restore the structure
to such form. Because some operation may corrupt this normalized shape we
call the \Restore{} operation at beginning of all other operations.

To allow the user to modify user functions (\Split, \Join, \dots) for each
operation we will make the \Restore{} callable by the user independently.

In the second implementation we will use this functionality to turn off
expensive updates during the \Expose{} operation, \Restore{} the structure back to the
correct form and before next operation turn the expensive updates back on.

\subsection{\sc Cut}

Operation $\Cut(u,v)$ deletes edge between vertices $u$ and $v$ and reorganizes
the Top Trees structure to reflect this change. Precondition for this operation
is that $u\ne v$ and there exists edge $(u,v)$.

Both implementations use different approaches, but the result is the same --
they remove the $(u,v)$ edge and return roots of two new top trees.

\subsection{\sc Link}

The Link operation is an opposite to the \Cut{} operation. Calling $\Link(u,v)$ on
two disconnected vertices joins them by the new edge $(u,v)$. Precondition for
this operation is that $u$ and $v$ are disconnected.

Both implementations use different approaches, but the result is the same --
both top trees are joined by the new edge $(u,v)$ and the cluster of resulting
top tree is returned.

\subsection{\sc Search}

When defined the \Choose{} user function this operation could be used to find
and return specific base cluster.

\todo{Example of search.}
