\chapter{Experiments}
\label{chap:Experiments}

Comparison of both implementations is an important part of the whole thesis
and for objective results multiple tests were needed.

I choose two different problems mentioned in the \cref{chap:Problems}.
First of them is a problem of maximum edge weight between given vertices with
interval update (described in section \ref{sec:maximum_edge_weight}), which uses
Top trees directly and aims to work in time $\O(\log N)$ per operation. Second
problem is an edge 2-connectivity (described in section \ref{sec:edge_2_connectivity}),
which uses Top trees ``under the hood'' and aims to work in time $\O(\log^4 N)$
per operation. Both problems uses Top trees in different ways.

Another necessary condition of good comparison is to have various input data. In
our case of investigating Top trees behaviour with given problems input data
consist of two things:
\begin{itemize}
\item Size and type of the underlying graph (number of edges, degrees of vertices)
\item Strategy of data structure usage (portion of affected edges, ration of
operation types, \dots)
\end{itemize}

\section{Experiments strategy}

Common scenario for all experiments were introduced. Each experiment was done
for both implementations with the same input data and with increasing input
size:

\begin{itemize}
\item Choose input size sampling
\item For each chosen input size choose 10 random seeds to generate input data
and sequence of operations
\item For every implementation and every seed run test and measure elapsed time
\item Compute average elapsed time for each implementation and save it
\end{itemize}

Python wrapper was used to generate random seeds and to execute testing utilities.
Testing utilities written in \Cpp{} firstly generates all the input data and
sequence of operations, initializes data structures and then began to measure
time and execute the operations.

This procedure was chosen to minimize influence of test functions and to measure
only the time used by Top trees implementations.

\section{Maximum edge weight experiments}

\subsection{Experiment 1}

\subsection{Experiment 2}

\section{Edge 2-connectivity experiments}

\subsection{Experiment 1}

\subsection{Experiment 2}
