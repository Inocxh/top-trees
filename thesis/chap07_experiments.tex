\chapter{Experiments}
\label{chap:Experiments}

Comparison of both implementations is an important part of the whole thesis
and for objective results multiple tests were needed.

I choose two different problems mentioned in the \cref{chap:Problems}.
First of them is a problem of maximum edge weight between given vertices with
interval update (described in section \ref{sec:maximum_edge_weight}), which uses
Top trees directly and aims to work in time $\O(\log N)$ per operation. Second
problem is an edge 2-connectivity (described in section \ref{sec:edge_2_connectivity}),
which uses Top trees ``under the hood'' and aims to work in time $\O(\log^4 N)$
per operation. Both problems uses Top trees in different ways.

Another necessary condition of good comparison is to have various input data. In
our case of investigating Top trees behavior with given problems input data
consist of two things:
\begin{itemize}
\item Size and type of the underlying graph (number of edges, degrees of vertices)
\item Strategy of data structure usage (portion of affected edges, ration of
operation types, \dots)
\end{itemize}

\section{Experiments strategy}

Common scenario for all experiments was introduced. Each experiment was done
for both implementations with the same input data and with increasing input
size:

\begin{itemize}
\item Choose graph size (number of vertices and edges)
\item For each chosen graph size choose 10 random seeds to generate input data
and sequence of operations
\item For every implementation and every seed run test and measure elapsed time
\item Compute average elapsed time and standard deviation for each input size
\end{itemize}

Python wrapper was used to generate random seeds and to execute testing utilities.
Testing utilities written in \Cpp{} firstly generates all the input data and
sequence of operations, initializes data structures and then began to measure
time and execute the operations.

This procedure was chosen to minimize influence of test functions and to measure
only the time used by Top trees operations.

\vfill\eject %% PRINTHACK

\section{Maximum edge weight experiment}
\label{sec:experiment_maximum_edge_weight}

This problem was described in section \ref{sec:maximum_edge_weight} and it
operates directly on the underlying tree.

For given size $N$ of the tree and number of operations the initial tree with
$N$ edges and $N+1$ vertices was generated. Number of edges during execution was
maintained between $N$ and $\frac{7}{10}N$ to maintain the input size.

After that the list of operations was generated (every operation with the same
chance). Types of operations are:

\begin{itemize}
\item Add edge -- Choose random two vertices and execute \Link{} operation
(when we choose two vertices in the same top tree the Top trees structure should
return an error).
\item Remove edge -- If there are at least $\frac{7}{10}N$ edges choose random
edge from list of edges and execute \Cut{} operation (we will maintain array of
added edges so every remove operation should be successful), otherwise skip this
operation.
\item Add weight on path -- Execute \Expose{} operation and modify content of
the returned root cluster.
\item Get weight on path -- Execute \Expose{} operation and read content of the
returned root cluster.
\end{itemize}

Every operation executes $\O(1)$ top trees operations and it should take time
$\O(\log N)$ (where $N$ is a number of edges).

Both implementations were tested on the same input data. For each implementation
the initial edges and list of operations were generated from the same random
seed. Then the implementation was initialized from the initial edges and time of
this initialization was measured. And finally all the operations were executed
and running time of all operations was measured.

Results from this experiment follows in the section
\ref{sec:results_maximum_edge_weight}.

Source code of the library for maximum edge weight is located in the header file
\texttt{include/examples/maximum\_edge\_weight.hpp} and source code of the
experiment itself is in the \texttt{src/experiment\_edge\_weight.cpp} file.


\vfill\eject %% PRINTHACK

\section{Edge 2-connectivity experiment}
\label{sec:experiment_edge_2_connectivity}

This problem was described in the section \ref{sec:edge_2_connectivity} and it
uses top trees as helper data structure for more complex operations.

Cover informations with incident edges are stored in top clusters and as a
supplement to the top trees structure it holds informations about all nontree
edges and some global counters.

\subsection{Stored data}

Information stored in each top cluster:
\begin{itemize}
\item Cover level of the cluster (maximal level on which is the cluster path
covered), cover edge of this level.

\item Cover informations to be distributed into children on splitting (lazy
propagation of {\it cover} and {\it uncover} operations).

\item Arrays with incident information for both endpoints of this cluster, size
of the arrays are $\O(\log^2 N)$.
\end{itemize}

Outside of the top clusters there are structures associated to the each edge
(because edge may change from tree to nontree and vice versa we need to associate
this structure with each edge) which holds information about level of the edge
and if it is coverage by other edges (like the cover information in top
clusters). Each edge have $\O(1)$ informations associated with it.

And finally there are global incident counters for each vertex, they takes
$\O(\log^2 N)$ space.

Because there are a lot of other data structures outside of the top trees
structure we cannot use the top trees structure initialization from underlying
base graph, but we must initialize the structure by inserting all the edges one
by one. During experiments the initialization time was measured independently.

\subsection{Generating initial graph and choosing number of edges}

Important decision during generating graph was number of edges. For graph on $N$
vertices we could have up to $\O(N^2)$ edges, but such graph wouldn't be much
interesting for the question of edge 2-connectivity.

We want to have graphs which are not fully 2-connected, but whose have large
2-connected components. During testing graph with sizes $cN^2$, $N\sqrt{N}$,
$N\log N$ and $cN$ were tested (where $N$ is number of vertices and $c$ some
small constant) and size $N\log N$ was chosen as an optimal size with the best
ratio of positive and negative queries (each other size was more shifted to some
side).

During experiment the number of edges of the underlying graph was kept between
$\frac{7}{10}N\log N$ and $\frac{13}{10}N\log N$.

\subsection{Test scenarios}

Possible operations were described in the section \ref{sec:edge_2_connectivity}.
For measuring time two scenarios were introduced. In the first one 70 \% of
operations were queries, 15 \% of them insertions and last 15 \% were deletions.
This experiment aims model average structure usage with all used operations.

Second scenario does only queries to properly measure difference between normal
variant of the second implementation and with variant of the second
implementation with disabled expensive updates during queries.

Results from this experiment follows in the section \ref{sec:results_edge_2_connectivity}.

Source code of the library for maximum edge weight is located in the header file
\texttt{include/examples/double\_edge\_connectivity.hpp} and source code of the
experiment itself is in the \texttt{src/experiment\_double\_edge\_connectivity.cpp} file.

\bigskip

The problem of edge 2-connectivity and its solution is much larger problem than
the problem in the first experiment. I tried to implement it in the best way
but my target was the comparison of the two top tress implementations, not
building universal solution for edge 2-connectivity.

The experiment still has some undefined behavior in some edge cases (for some
random seeds), but it has non effect on comparison -- when some failure occurred
another random seed was chosen.
